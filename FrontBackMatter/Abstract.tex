%-------------------------------------------------------------------------------
%	ABSTRACT PAGE
%-------------------------------------------------------------------------------

\addchaptertocentry{Хураангуй} % Хураангуйг гарчигт нэмэх

\begin{center}
{\scshape\Large \univname\par} % Их сургуулийн нэр
{\scshape\large \facname\par}\vspace{0.5cm} % Их сургуулийн нэр
{\huge\textbf{{Хураангуй}} \par}
\bigskip
{\Large{\ttitle} \par} % Тезисийн нэр
\bigskip

{\normalsize \shortname \par} % Зохиогчийн нэр
\addressname
\end{center}

\textit{\textbf{Түлхүүр үгс: \keywordnames}}
\bigskip

Энэхүү төгсөлтийн ажлын хүрээнд хэрэглэгч өөрийн өгөгдлийг бүрэн хянах боломжтой, нөөц багатай төхөөрөмж (Raspberry Pi) дээр ажиллах хувийн үүлэн хадгалалтын системийг зохиомжлон хэрэгжүүлэв. Уг систем нь гуравдагч этгээдийн хадгалалтын үйлчилгээнээс хамааралгүйгээр өгөгдлийн өмчлөл, нууцлалыг хангах зорилготой. Системийн үндсэн архитектур нь дотоод сүлжээнд серверийг автоматаар илрүүлэх болон шууд холболт боломжгүй үед дамжуулагч сервер (Relay Server) ашиглан интернэт орчинд автомат холболт тогтоох шийдлүүдээс бүрдэнэ

Хөгжүүлэлтийн явцад олон хэрэглэгчийн зэрэг холболтыг дэмжих зэрэгцээ боловсруулалт бүхий дамжуулагч серверийн архитектурыг зохиож , хэрэглэгчийн гар утасны (IOS) программыг боловсруулсан. Аюулгүй байдлыг хангах үүднээс бүх өгөгдөл дамжуулалтад End-to-End шифрлэлт болон QR кодод суурилсан холболтын механизмыг ашигласан болно.



