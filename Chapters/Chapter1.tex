% Бүлэг 1

\chapter{Хувийн үүлэн хадгалалтын системийн тухай онол, арга зүйн судалгаа} % Бүлгийн нэр
\label{Chapter1} % Энэ бүлэг рүү ишлэл хийх бол \ref{Chapter1} командыг ашигла 

%-------------------------------------------------------------------------------

% Агуулгад ашигласан хэвшүүлэлтийн зарим командын тодорхойлолт
\newcommand{\keyword}[1]{\textbf{#1}}
\newcommand{\tabhead}[1]{\textbf{#1}}
\newcommand{\code}[1]{\texttt{#1}}
\newcommand{\file}[1]{\texttt{\bfseries#1}}
\newcommand{\option}[1]{\texttt{\itshape#1}}

%-------------------------------------------------------------------------------



%-------------------------------------------------------------------------------
\section{Ижил төст системийн судалгаа}
\paragraph{}
Хувийн өгөгдөл хадгалах, синхрончлох системүүд нь сүүлийн жилүүдэд хэрэглэгчийн өгөгдөлд бүрэн хяналттай байх, аюулгүй байдал болон cloud-ын энгийн хэрэглээний туршлагыг хадгалах чиглэлд эрчимтэй хөгжиж байна. Энэ хүрээнд уламжлалт төвлөрсөн cloud системүүд болон peer-to-peer системүүдийг судалгаанд оруулж байна.
\paragraph{}
Nextcloud нь self-hosted cloud storage системүүдийн түгээмэл жишээ бөгөөд сервер төвтэй архитектур ашиглан файл хадгалалт, синхрончлол, хуваалцах болон хамтын ажиллагааны олон төрлийн боломжуудыг нэг дор нэгтгэдэг. Nextcloud нь Universal File Access \cite{nextcloud2018architecture} давхаргаар дамжуулан төрөл бүрийн storage эх үүсвэрүүдийг нэг интерфэйс дор нэгтгэх боломж олгодог. Гэсэн хэдий ч ийм өргөн хүрээтэй боломжууд нь системийг суулгах, тохируулах, удирдах процессыг төвөгтэй болгож, энгийн хэрэглэгчдэд хүндрэл учруулдаг.

\begin{figure}[H]
    \centering
    \includegraphics[scale=0.75]{Figures/chapter1/nextcloud_data_diagram.png}
    \caption{NextCloud системийн өгөгдлийн давхаргын нэгдсэн шийдэл}
    \label{fig:nextcloudDataDiagram}
\end{figure}

\paragraph{}
Нөгөө талаас Syncthing нь peer-to-peer зарчимд суурилсан файл синхрончлолын систем бөгөөд төв сервер шаардалгүйгээр төхөөрөмжүүдийг хооронд нь шууд холбодог. Syncthing нь Local Discovery, Global Discovery, Relay Protocol зэрэг механизмуудыг \cite{syncthingspecifications} ашиглан сүлжээний янз бүрийн орчинд төхөөрөмжүүдийг илрүүлэх, холбох боломжийг олгодог. Мөн бүх өгөгдөл end-to-end encryption (E2EE)-ээр\ref{sec:e2ee} хамгаалагддаг нь аюулгүй байдлын чухал давуу тал юм. Гэвч Syncthing нь олон төхөөрөмж дээр нэг фолдерийг хуулбарлан хадгалах (replication) зарчимд суурилсан тул уламжлалт cloud storage-ийн өгөгдөл нэг газарт төвлөрсөн байдлаар хадгалагдана гэсэн ойлголттой нийцдэггүй. 
\paragraph{}
Мөн хэрэглэгчийн хүлээлтийг тодорхойлогч consumer cloud storage системүүд болох Apple iCloud болон Google Drive нь автомат backup, background upload, минимал тохиргоо зэрэг UX-ийн өндөр стандартыг тогтоосон. Эдгээр системүүд нь хэрэглээний хувьд энгийн боловч өгөгдөл нь гуравдагч талын төв серверт хадгалагддаг.

\subsection{Нийтийн үүлэн хадгалалтын системүүд}
Нийтийн үүлэн хадгалалтын системүүдэд дараах өргөн хэрэглэгддэг платформууд орно:

\begin{enumerate}
    \item Apple iCloud
    \item Google Drive
    \item Microsoft OneDrive
\end{enumerate}

\paragraph{}
Нийтийн үүлэн хадгалалтын системүүд нь автомат синхрончлол, нөөцлөлт зэрэг үйлдлүүдийг хэрэглэгчийн оролцоогүйгээр гүйцэтгэх замаар өндөр түвшний хэрэглээний энгийн байдлыг хангадаг. Гэсэн хэдий ч өгөгдөл нь гуравдагч талын төв серверт хадгалагддаг тул хэрэглэгчийн өгөгдөлд бүрэн хяналттай байх боломж хязгаарлагддаг.

\begin{figure}[H]
    \centering
    \includegraphics[scale=0.75]{Figures/chapter1/cloud_providers.png}
    \caption{Өргөн ашиглагдаж буй үүлэн хадгалалтын үйлчилгээнүүд}
    \label{fig:popularCloudProviders}
\end{figure}


\subsection{Хувийн үүлэн хадгалалтын системүүд}
Self-hosted cloud системүүдийн түгээмэл жишээнүүд:
\begin{enumerate}
    \item NextCloud (opensource)
    \item OwnCloud (opensource)
    \item Seafile (opensource)
\end{enumerate}

\paragraph{}
Self-hosted cloud системүүд нь уламжлалт сервер төвтэй архитектур дээр суурилан ажилладаг бөгөөд хэрэглэгчийн өгөгдлийг өөрийн хяналт дор хадгалах боломжийг олгодог. Эдгээр системүүд нь хэрэглэгчийн интерфэйсийн хувьд харьцангуй боловсронгуй боловч сүлжээний орчин, серверийн тохиргоо, удирдлагын нэмэлт мэдлэг шаарддаг. Иймээс уг төрлийн шийдлүүд нь байгууллагын түвшинд хамтран ашиглахад илүү тохиромжтой бөгөөд хувийн хэрэглээнд ашиглахад техникийн хүндрэл үүсгэх магадлалтай.

\begin{figure}[H]
    \centering
    \includegraphics[scale=0.75]{Figures/chapter1/self_cloud_providers.png}
    \caption{Хувийн сервер дээр ашиглах боломжтой үүлэн хадгалалтын платформууд}
    \label{fig:popularCloudProviders}
\end{figure}


\subsection{Peer-to-Peer өгөгдөл синхрончлолын системүүд}

Peer-to-peer (P2P) зарчимд суурилсан өгөгдөл синхрончлолын системүүд:
\begin{enumerate}
    \item Syncthing (opensource)
    \item Resilio Sync
\end{enumerate}

Эдгээр системүүд нь төхөөрөмж бүрийг давтагдашгүй таних тэмдэг (device identity)-ээр тодорхойлж, төв сервер шаардахгүйгээр хооронд нь шууд холбох боломжийг олгодог. Төхөөрөмжүүдийг илрүүлэхдээ дотоод сүлжээнд LAN discovery механизмыг ашиглах бөгөөд шаардлагатай тохиолдолд relay fallback \cite{syncthingspecifications} шийдлийг ашиглан холболтыг үргэлжлүүлэх боломжтой. Мөн өгөгдөл дамжуулалт нь end-to-end encryption (E2EE)-д суурилсан zero-trust загвараар хамгаалагддаг нь аюулгүй байдлын чухал давуу тал болдог.

Гэсэн хэдий ч эдгээр системүүд нь төхөөрөмжүүдийн хооронд файлыг харилцан хуулбарлан хадгалах (replication) зарчимд суурилдаг тул өгөгдөл нэг төв байршилд хадгалагдах уламжлалт хувийн cloud storage загвараас ялгаатай. Иймээс тэдгээрийн хэрэглээний зорилго нь бүрэн нийцэхгүй боловч дотооддоо ашиглаж буй холболт, илрүүлэлт болон аюулгүй дамжуулалтын технологиуд нь судлан хэрэгжүүлэхэд чухал ач холбогдолтой юм.

\subsection{Харьцуулсан шинжилгээ}

\paragraph{}
Дараах хүснэгтээс харахад одоо байгаа шийдлүүд нь хэрэглээний энгийн байдал, өгөгдлийн хяналт болон сүлжээний уян хатан байдлын хооронд тодорхой зөрчил (trade-off) үүсгэж байна. Нийтийн үүлэн системүүд нь хэрэглээний хувьд энгийн боловч хэрэглэгчийн өгөгдөлд бүрэн хяналт олгодоггүй. Харин self-hosted болон P2P системүүд нь өгөгдлийн хяналт болон аюулгүй байдлыг сайжруулдаг ч суурилуулалт болон ашиглалтын төвөгшил нэмэгддэг. Иймээс эдгээр системүүд нь хэрэглээний энгийн байдал, аюулгүй байдал болон сүлжээний уян хатан байдлыг зэрэг хангах нэгдсэн шийдлийг бүрэн санал болгож чадахгүй байна.

\begin{table}[H]
\centering
\begin{tabular}{|L{0.18\textwidth}|L{0.18\textwidth}|C{0.13\textwidth}|C{0.13\textwidth}|C{0.13\textwidth}|C{0.13\textwidth}|}
\hline
\rowcolor{gray!10}
\textbf{Систем} & \textbf{Архитектур} & \textbf{LAN Auto Discovery} \ref{sec:lan-detection} & \textbf{Relay Fallback} & \textbf{E2EE} \ref{sec:e2ee} & \textbf{UX энгийн байдал} \\ 
\hline
Nextcloud & Төвлөрсөн self-hosted сервер & Байхгүй & Байхгүй & Хэсэгчлэн & Дунд \\ 
\hline
Syncthing & Peer-to-Peer (P2P) & Байгаа & Байгаа & Байгаа & Харьцангуй төвөгтэй \\ 
\hline
iCloud / Google Drive & Төвлөрсөн нийтийн cloud & Байхгүй & Байхгүй & Хязгаар -лагдмал & Өндөр \\ 
\hline
\rowcolor{green!20}
Санал болгож буй систем & Hybrid (Server + Relay) & Байгаа & Байгаа & Байгаа & Өндөр \\ 
\hline
\end{tabular}
\caption{Ижил төст системүүдийн харьцуулалт}
\end{table}

\section{End-to-End шифрлэлт (E2EE)}
\label{sec:e2ee}

\paragraph{}
End-to-End шифрлэлт гэдэг нь нэг цэгээс нөгөө цэгд өгөгдлийг дамжуулахдаа шифрлэж илгээх процессийг хэлдэг. Өгөгдөл нь илгээгдэх үедээ шифрлэлттэй илгээгдэж хүлээн авагч нь л тайлж уншдаг. Мессеж чат бичих аппууд болон бусад харилцаа холбооны үйлчилгээнүүд E2EE-г ашиглаж мессежүүдийг зөвшөөрөлгүй хандалтаас сэргийлдэг. сүлжээгээр харилцах хамгийн аюулгүй арга гэж тооцогддог. \cite{ibm-e2ee}

\paragraph{}
E2EE нь бусад шифрлэлтийн аргуудаас ялгаатай нь өгөгдлийг эхлэлээс төгсгөл хүртэл хамгаалдгаараа онцлог юм. Энэ нь өгөгдлийг илгээгчийн төхөөрөмж дээр шифрлэж, дамжуулалтын турш шифрлэгдсэн хэвээр хадгалан, зөвхөн хүлээн авагчийн төхөөрөмж дээр тайлдаг. Ийнхүү дамжуулалтыг зуучилж буй үйлчилгээ үзүүлэгчид, жишээлбэл WhatsApp зэрэг системүүд, дамжуулж буй мэдээллийн агуулгад нэвтрэх боломжгүй болдог. Өөрөөр хэлбэл, зөвхөн илгээгч болон зорилтот хүлээн авагч л мэдээллийг унших боломжтой байдаг.

\paragraph{}
Үүнтэй харьцуулахад дамжуулалтын үеийн шифрлэлт (encryption in transit) нь өгөгдлийг зөвхөн дамжиж буй хугацаанд хамгаалдаг. Жишээлбэл, Transport Layer Security (TLS) протокол нь өгөгдлийг клиент болон серверийн хооронд дамжих үед шифрлэдэг. Гэвч энэ нь өгөгдлийг дамжуулалтын дараах шатанд хамгаалахгүй бөгөөд програмын серверүүд эсвэл сүлжээний үйлчилгээ үзүүлэгчид зэрэг завсрын оролцогчид өгөгдөлд нэвтрэх боломжтой хэвээр байдаг.

\subsection{E2EE хэрхэн ажилладаг вэ?}

End-to-End шифрлэлтийн процесс нь уншигдах боломжтой өгөгдлийг уншигдах боломжгүй хэлбэрт хувиргаж, аюулгүй байдлаар дамжуулан, хүрэх цэг дээр нь дахин анхны хэлбэрт нь сэргээх үйл явцыг агуулдаг.

\textbf{Нарийвчилбал E2EE дараах үндсэн 4 хэсгийг агуулдаг:} \cite{ibm-e2ee}

\begin{enumerate}
    \item Шифрлэлт
    \item Дамжуулалт
    \item Өгөгдлийг тайлах
    \item Баталгаажуулалт
\end{enumerate}

\subsubsection{Шифрлэлт}

\paragraph{}
E2EE нь нууц өгөгдлийг шифрлэх алгоритмыг ашигласнаар эхэлдэг. Энэхүү алгоритм нь нарийн төвөгтэй математик функцуудыг ашиглан өгөгдлийг уншигдах боломжгүй хэлбэрт хувиргадаг бөгөөд үүнийг шифрлэгдсэн текст (ciphertext) гэж нэрлэдэг. Зөвхөн тайлах түлхүүр (decryption key) бүхий эрх бүхий хэрэглэгчид л уг мэдээллийг унших боломжтой байдаг.
\paragraph{}
E2EE нь өгөгдлийг шифрлэх болон тайлахад хоёр өөр түлхүүр ашигладаг тэгш хэмт бус (asymmetric) шифрлэлтийн схемийг, эсвэл нэг ижил нууц түлхүүр ашигладаг тэгш хэмт (symmetric) шифрлэлтийн схемийг ашиглаж болно. Ихэнх E2EE хэрэгжилтүүд нь эдгээр хоёр аргыг хослуулан ашигладаг. \ref{sec:e2ee-encryption} хэсэгт дэлгэрүүлж бичсэн байгаа.

\subsubsection{Дамжуулалт}

\paragraph{}
Шифрлэгдсэн өгөгдөл (ciphertext) нь интернет болон бусад сүлжээ зэрэг харилцаа холбооны сувгаар дамжин зорьсон газартаа хүрдэг. Дамжуулалтын явцад уг мэдээлэл нь програмын серверүүд, интернет үйлчилгээ үзүүлэгчид (ISP), халдагчид болон бусад этгээдүүдэд уншигдах боломжгүй хэвээр байна. Хэрэв дамжуулалтын үеэр хэн нэгэн мэдээллийг барьж авсан тохиолдолд энэ нь санамсаргүй, ойлгомжгүй тэмдэгтүүдийн дараалал мэт харагдах болно.

\subsubsection{Өгөгдлийг тайлах}

\paragraph{}
Шифрлэгдсэн өгөгдөл хүлээн авагчийн төхөөрөмжид хүрэхэд асимметрик шифрлэлтийн үед хүлээн авагчийн хувийн түлхүүр, харин симметрик шифрлэлтийн үед урьдчилан хуваалцсан нууц түлхүүрийг ашиглан тайлагдана. Өгөгдлийг тайлахад шаардлагатай хувийн түлхүүрийг зөвхөн хүлээн авагч эзэмшдэг.

\subsubsection{Баталгаажуулалт}

\paragraph{}
Тайлсан өгөгдлийн бүрэн бүтэн байдал болон жинхэнэ эх сурвалжийг баталгаажуулах зорилгоор шалгалт хийгддэг. Энэ шатанд илгээгчийн дижитал гарын үсэг эсвэл бусад баталгаажуулах мэдээллийг шалгаж, дамжуулалтын явцад өгөгдөлд ямар нэгэн өөрчлөлт орсон эсэхийг тодорхойлно.

\subsection{Шифрлэлтийн алгоритм ба тэгш хэмт ба тэгш хэмт бус шифрлэлт (Symmetric vs asymmetric)}
\label{sec:e2ee-encryption}
\paragraph{}
Шифрлэлтийн үндсэн хоёр арга болох тэгш хэмт (symmetric) болон тэгш хэмт бус (asymmetric) шифрлэлт нь нууц түлхүүрийг ашиглах зарчмаараа ялгаатай. Тэгш хэмт шифрлэлт нь өгөгдлийг шифрлэх болон тайлахад нэг ижил нууц түлхүүр ашигладаг бөгөөд хурд болон үр ашгийн хувьд давуу талтай боловч түлхүүрийн аюулгүй удирдлагыг шаарддаг. Харин тэгш хэмт бус шифрлэлт нь хоёр өөр криптограф түлхүүр болох нийтийн түлхүүр (public key) болон хувийн түлхүүр (private key)-ийг ашигладаг бөгөөд түлхүүрийг аюулгүйгээр хуваалцах асуудлыг шийдвэрлэдэг боловч боловсруулах хурд харьцангуй удаан байдаг.

\begin{itemize}
    \item \textbf{Тэгш хэмт шифрлэлт (Symmetric Encryption):} Өгөгдөл $M$-ийг $K$ түлхүүрээр шифрлэх ($E$) ба тайлах ($D$) процесс нь ижил түлхүүр ашиглана:
    \begin{equation}
        C = E_K(M), \quad M = D_K(C)
    \end{equation}
    \item \textbf{Тэгш хэмт бус шифрлэлт (Asymmetric Encryption):} Илгээгч нь хүлээн авагчийн нийтийн түлхүүр $K_{pub}$-аар шифрлэх ба зөвхөн хүлээн авагч өөрийн хувийн түлхүүр $K_{priv}$-ээр тайлна:
    \begin{equation}
        C = E_{K_{pub}}(M), \quad M = D_{K_{priv}}(C)
    \end{equation}
\end{itemize}

\paragraph{}
Дээрх онолын аргуудыг бодит системд хэрэгжүүлэхдээ гүйцэтгэл болон аюулгүй байдлын тэнцвэрийг хадгалах дараах алгоритмуудыг ашиглах нь тохиромжтой байна:
\begin{enumerate}
\item \textbf{AES-GCM (Advanced Encryption Standard - Galois/Counter Mode):}
\textit{Төрөл: Тэгш хэмт (Symmetric).}
Өнөөдрийн байдлаар дэлхийн хамгийн найдвартай стандарт юм. Гүйцэтгэлийн хувьд Raspberry Pi 4 болон орчин үеийн ухаалаг утасны процессорууд дахь техник хангамжийн хурдасгуурыг (hardware acceleration) ашигладаг тул маш хурдан ажилладаг. Мөн өгөгдлийг шифрлэхийн зэрэгцээ түүний бүрэн бүтэн байдлыг (integrity) шалгах AEAD горимыг дэмждэг.

\item \textbf{ChaCha20-Poly1305:} 
\textit{Төрөл: Тэгш хэмт (Symmetric).} 
Google-ийн санал болгосон алгоритм бөгөөд техник хангамжийн хурдасгуургүй бага хүчин чадалтай төхөөрөмжүүд дээр AES-ээс илүү хурдан ажилладаг. Программ хангамжийн түвшинд хэрэгжүүлэхэд хялбар, аюулгүй байдлын хувьд AES-тэй ижил түвшинд үнэлэгддэг тул мобайл аппликейшнд өргөн ашигладаг.

\item \textbf{X25519 (Curve25519):} 
\textit{Төрөл: Тэгш хэмт бус (Asymmetric / Key Exchange).} 
Төхөөрөмжүүд хоорондоо нууц түлхүүрээ аюулгүй солилцоход ашигладаг Elliptic Curve Diffie-Hellman (ECDH) протоколын хамгийн түгээмэл хувилбар юм. RSA-тай харьцуулахад түлхүүрийн хэмжээ маш бага (256-бит) боловч аюулгүй байдал нь өндөр тул сүлжээний зурвасын өргөнийг хэмнэдэг.

\end{enumerate}

\paragraph{}
Иймээс End-to-End Encryption (E2EE)-ийг хэрэгжүүлдэг системүүд ихэвчлэн эдгээр хоёр аргыг хослуулан ашигладаг. Жишээлбэл, хэрэглэгчид хооронд харилцаа эхлэх үед тухайн session-д зориулсан өвөрмөц session түлхүүр үүсгэж, мессежүүдийг symmetric аргаар шифрлэдэг. Энэхүү session түлхүүрийг asymmetric шифрлэлтийн аргаар дамжуулан хуваалцдаг бөгөөд хүлээн авагчийн нийтийн түлхүүрээр шифрлэгдэж, зөвхөн түүний хувийн түлхүүрээр тайлагддаг. Ингэснээр систем нь asymmetric шифрлэлтийн аюулгүй байдлыг symmetric шифрлэлтийн өндөр үр ашигтай хослуулсан хамгаалалтын механизмыг бүрдүүлдэг.

\subsection{E2EE-ийн нийтлэг хэрэглээний тохиолдлууд}

E2EE-ийн нийтлэг хэрэглээнд дараах хувийн эмзэг мэдээлэл хадгалах хэрэглээнүүд ордог:

\begin{itemize}
    \item \textbf{Найдвартай харилцаа холбоо:} Apple iMessage, WhatsApp аппууд нь E2EE ашиглаж хэрэглэгчдийн хооронд мессеж илгээдэг. 
    \item \textbf{Нууц үгийн зохицуулга:} 1Password, Bitwarden, Dashlane, LastPass гэх мэт хэрэглэгчийн нууц үг зохицуулгын үйлчилгээнүүд хэрэглэгчийн төхөөрөмж хооронд E2EE ашиглан нууц үгийг илгээдэг.
    \item \textbf{Файл хуваалцах:} E2EE нь өгөгдлийг дамжуулах явцад зөвшөөрөлгүй этгээдүүд нэвтрэхээс хамгаалдаг бөгөөд P2P файл солилцоо, шифрлэгдсэн cloud хадгалалт болон хамгаалагдсан файл дамжуулах үйлчилгээнд өргөн ашиглагддаг.
\end{itemize}

\paragraph{}
Хувийн үүлэн хадгалалтын системийн аюулгүй байдлын үндсэн зарчим нь хэрэглэгчийн мобил төхөөрөмж болон хувийн сервер хооронд дамжих бүх өгөгдлийг \textit{End-to-End Encryption (E2EE)} аргаар хамгаалахад оршино. Систем нь сүлжээний NAT болон галт ханын хязгаарлалтыг давахын тулд дамжуулагч сервер (\textit{Relay Server}) ашиглах бөгөөд энэ нөхцөлд өгөгдөл замаас задрахгүй байх нь нэн чухал юм. Иймд \textit{No-Trust} буюу үл итгэх загварыг баримтлан, дамжуулагч серверт өгөгдлийг тайлах ямар ч боломж олгохгүйгээр зөвхөн шифрлэгдсэн пакетуудыг дамжуулах бөгөөд энэ нь системийн нууцлалыг хамгийн дээд түвшинд хангана.

\section{Service Detection in the LAN}
\label{sec:lan-detection}

\section{Raspberry Pi-д зориулсан программ хангамж}
\label{sec:raspberrypi}

\section{Ашиглах технелогиуд}


\section{Бүлгийн дүгнэлт}
%-------------------------------------------------------------------------------
\paragraph{} 