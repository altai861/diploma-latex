% Бүлэг 1

\chapter{... системийн тухай онол, арга зүйн судалгаа} % Бүлгийн нэр
\label{Chapter1} % Энэ бүлэг рүү ишлэл хийх бол \ref{Chapter1} командыг ашигла 

%-------------------------------------------------------------------------------

% Агуулгад ашигласан хэвшүүлэлтийн зарим командын тодорхойлолт
\newcommand{\keyword}[1]{\textbf{#1}}
\newcommand{\tabhead}[1]{\textbf{#1}}
\newcommand{\code}[1]{\texttt{#1}}
\newcommand{\file}[1]{\texttt{\bfseries#1}}
\newcommand{\option}[1]{\texttt{\itshape#1}}

%-------------------------------------------------------------------------------



%-------------------------------------------------------------------------------
\section{... тухай}
\paragraph{}
Компьютерын хараа /Computer Vision/ бол визуал дата /Visual Data/ буюу зураг болон видео файлтай ажиллаж, тэдгээр дээр боловсруулалт, анализ хийн, компьютерыг зураг, видеог хүн шиг ойлгодог болгох тал дээр төвлөрдөг компьютерын шинжлэх ухааны салбар шинжлэх ухаан. Харин объект илрүүлэлт /Object Detection/ нь зураг эсвэл видео доторх объектуудыг олох, танихад чиглэгддэг компьютерын харааны техник юм. Тодруулбал, объект илрүүлэлт нь эдгээр илрүүлсэн объектуудаа хүрээлсэн хайрцаг зурдаг бөгөөд энэ нь тухайн объектууд нь яг юу болохыг тодорхойлж, тухайн үзэгдэлд хаана байгааг (эсвэл тэд хэрхэн хөдөлж байгааг) олох боломжийг олгодог \cite{medeeStatistic}. Объект илрүүлэлт нь ихэвчлэн зургийн танилт /Image Recognition/ -тай андуурагддаг. Тиймээс объект илрүүлэлтийг цааш судлахаасаа өмнө тэдгээрийн хоорондын ялгааг тодруулах нь чухал юм. Зураг \ref{fig:regVSdec} -д зураг танилт болон объект илрүүлэлтийн ялгааг харуулсан зургийг харуулсан.
\begin{figure}[H]
    \centering
    \includegraphics[scale=0.85]{Figures/chapter1/imgRegVSobjDec.PNG}
    \caption{Зураг танилт болон объект илрүүлэлтийн ялгаа}
    \label{fig:regVSdec}
\end{figure}
\newline

%-------------------------------------------------------------------------------
\section{... бий болгосон эхэн үеийн алгоритмууд}
\paragraph{}Объект илрүүлэлтийн эрин үеийг үндсэн хоёр үед хуваан үзэж болно. Эхний үе нь “Уламжлалт объект илрүүлэлтийн үе” буюу 2014 оноос өмнөх үе, дараагийн үе нь “Гүний сургалт дээр суурилсан объект илрүүлэлтийн үе” буюу 2014 оноос хойших онууд \cite{Reference1}.  
\subsection{Виола Жонес илрүүлэлт}
\paragraph{}
Паул Виола болон Михаел Жонес нарын 2001 онд хөгжүүлсэн хүний нүүр царайг илрүүлэх арга. Энэ нь HAAR -төст онцлогуудыг зураг дээрээс олсноор хүний нүүр царайг таньдаг. Зураг \ref{fig:edge&lineFeatures} -д Виола Жонес илрүүлэлтэд ашигладаг HAAR төст онцлогуудыг дүрсэлсэн.
\begin{figure}[H]
    \centering
    \includegraphics[scale=0.85]{Figures/chapter1/edge&lineFeatures.PNG}
    \caption{HAAR төст онцлогууд}
    \label{fig:edge&lineFeatures}
\end{figure}
\par \noindent Зураг \ref{fig:featuresOnFace} -д хүний царайг дүрсэлсэн саарал зургаас HAAR -төст онцлогуудыг олсныг харуулж байна.
\begin{figure}[H]
    \centering
    \includegraphics[scale=0.85]{Figures/chapter1/featuresOnFace.PNG}
    \caption{HAAR ашигласан хүний царай илрүүлэлт}
    \label{fig:featuresOnFace}
\end{figure} \newline HAAR -төст онцлогууд буюу дараах HAAR гэж нэрлэгддэг цонхнуудыг(Анх Алфред Хаар гэх хүн энэ санааг дэвшүүлсэн учраас түүний нэрээр нэрлэсэн) саарал зураг/Grayscale Image/ дээгүүр гүйлгэж тухайн зургийн HAAR -тай төстэй хэсгийг онцлон авч хүний нүд, хөмсөг, уруул, хамрын хэсэг гэж үздэг байсан. Саарал зургийн хувьд пикселүүд нь зөвхөн хар болон цагаанаар ялгагдаж харагддаггүй учраас HAAR -тай төстэй хэсэг байна гэдгийг батлахын тулд дараах тооцооллыг хийдэг. 
Зураг дээгүүр гүйлгэж буй цонх буюу кернелийн 0,5 с доош утгатай байгаа пикселүүдийг нэмнэ. Мөн адил 0,5 с дээш утгатай байгаа пикселүүдийг хооронд нь нэмээд дараа нь тэдгээрийн зөрүүг олно. Зөрүү утга 1-тэй дөхөх тусам HAAR төст хэсэг байна гэж үзэн 1-тэй хамгийн ойр HAAR төст хэсгийг авна. Зураг \ref{fig:featureDetector} -д саарал зургийн пикселийг 0-1 хооронд дүрсэлснийг харуулж байна.
\begin{figure}[H]
    \centering
    \includegraphics[scale=1]{Figures/chapter1/featureDetector.PNG}
    \caption{Саарал зургийн дүрслэл}
    \label{fig:featureDetector}
\end{figure} 

 %\[ {\delta}= хар - цагаан =  \frac{1}{n} \sum_{хар}^{n} I(x) - \frac{1}{n} \sum_{цагаан}^{n} I(x) \] 
\subsection{HOG /Histogram of Oriented Gradients/ илрүүлэлт}
\paragraph{} 
Чиглэгдсэн градиентийг илрүүлэх гистограмм арга (HOG) нь бараг арван жилийн настай эртний алгоритмуудын нэг, гэсэн хэдий ч бусдаас ялгарах нэг зүйл бол өнөөг хүртэл маш их ашиглагдаж байгаа бөгөөд үр дүнтэй \cite{introHOG}. Чиглэгдсэн градиентийн гистограмм аргыг зургийг ангилахын тулд нүүр царай, дүрс илрүүлэхэд ашигладаг. Навнеет Далал, Билл Тригс нар 2005 онд чиглэгдсэн градиентийн гистограмм (HOG) функцийг нэвтрүүлсэн.
Чиглэгдсэн градиент тодорхойлогчийн гистограмын цаадах зарчим нь зураг доторх объектын дүр төрх, хэлбэрийг эрчмийн градиент эсвэл ирмэгийн чиглэлийн хуваарилалтаар тодорхойлж болно \cite{Reference2}. Зургийн x ба у градиентууд нь эрчимжилтийн огцом өөрчлөлтийн улмаас ирмэг ба булангийн эргэн тойронд градиентийн хэмжээ их байдаг тул ирмэг ба булангууд нь хавтгай хэсгүүдээс илүү объектын хэлбэрийн талаар илүү их мэдээлэл агуулдаг. Тиймээс градиентийн чиглэлийн гистограммыг энэ тодорхойлогчийн шинж чанар болгон ашигладаг. Зураг \ref{fig:stepOfHog}
-д HOG аргаар объект илрүүлэх алхмуудын дарааллыг харуулж байна.
\begin{figure}[H]
    \centering
    \includegraphics[scale=0.75]{Figures/chapter1/stephog.PNG}
    \caption{HOG аргаар объект илрүүлэх алхмууд}
    \label{fig:stepOfHog}
\end{figure}

 \par 
 HOG аргаар объект илрүүлэх алхмууд:
\begin{itemize}
  \item \textbf{Урьдчилан боловсруулах: }Зургийг урьдчилан боловсруулахын тулд нийтлэг харьцаа буюу (өргөн: өндөр) 1: 2 харьцаа руу зургийг хувиргах хэрэгтэй. 100x200, 500x1000 гэх мэт байж болно. Зураг \ref{fig:preprocessinghog} -д урьдчилсан боловсруулалтын үе шатанд явж буй хувиргагдсан харьцаатай зургуудыг дүрсэлж байна.
  \begin{figure}[H]
    \centering
    \includegraphics[scale=1]{Figures/chapter1/preprocessinghog.PNG}
    \caption{Урьдчилсан боловсруулалтын үе шат}
    \label{fig:preprocessinghog}
\end{figure}
  \item \textbf{Градиент тооцоолох: } Х болон Ү чиглэлд пиксел тус бүрд зургийн градиентийг тооцоолно. Зураг \ref{fig:calculateGradient} нь градиент тооцоолох томъёог харуулж байна.
  \begin{figure}[H]
    \centering
    \includegraphics[scale=1]{Figures/chapter1/calculateGradient.PNG}
    \caption{Градиент тооцоолох үе шат}
    \label{fig:calculateGradient}
\end{figure}
  \item \textbf{Градиентуудын гистограмм тооцоолох: }Энэ үйлдлийг ойлгомжтойгоор тайлбарлавал, зургийг хуваасан нүднүүдээ нэгтгэж 8х8 бол нийт 64 магнитуди үүсгээд тэдгээрийгээ өнцгөөс нь хамааруулан 9 ширхэг саванд хувааж хийнэ гэсэн үг. Зураг \ref{fig:gradientmagnitudeandirection} -д зургийн магнитудыг олсныг харуулж байна.
  \begin{figure}[H]
    \centering
    \includegraphics[scale=1]{Figures/chapter1/gradientmagnitudeandirection.PNG}
    \caption{Магнитуд болон чиглэл тодорхойлох}
    \label{fig:gradientmagnitudeandirection}
\end{figure}
\par Зураг \ref{fig:createvector} -д градиент чиглэл болон магнитуудыг ашиглан градиентуудын гистограмм үүсгэж байна.
 \begin{figure}[H]
    \centering
    \includegraphics[scale=1]{Figures/chapter1/createvector.PNG}
    \caption{Вектор үүсгэх}
    \label{fig:createvector}
\end{figure}
  \item \textbf{Блок нормалчлал: }Гэрэлтүүлгийн өөрчлөлт гистограммын хэмжээнд нөлөөлдөг. Тиймээс гистограммын утгыг гэрэлтүүлгээс хамааралгүй байлгахын тулд өмнөх алхамд үүсгэсэн гистограмм векторыг нормалчилдаг.
  \item \textbf{Вектор үүсгэх: }Үүсгэсэн векторын тодорхойлогч векторыг олно.
\end{itemize}
\par \noindent Зураг \ref{fig:visualhog} нь зургийн градиентуудыг дүрсэлсэн. Энэхүү дүрслэл нь градиентууд хаашаа шилжиж байгааг ойлгох, зургийн дотор объектууд хаана байгааг мэдэхэд хэрэгтэй байж болно.
\begin{figure}[H]
    \centering
    \includegraphics[scale=0.85]{Figures/chapter1/visualhog.PNG}
    \caption{HOG -н үр дүн}
    \label{fig:visualhog}
\end{figure}



\subsection{DPM /Deformable Part-based Model/ илрүүлэлт}
\paragraph{} Хэлбэрийг нь алдагдуулж болох хэсгүүдэд суурилсан загвар. Энэ нь 1973 оны Фишлер болон Элшлагер нарын "объектууд нь гажигтай хэсгүүдийн цуглуулга юм" гэсэн санаан дээрээс үүдэлтэй загвар \cite{introdDPM}. Хөгжүүлэлт нь HOG загвараас санаа авснаар эхэлсэн. DPM нь зургийг пирамид загвар руу шилжүүлэн, давхарга бүрд нь HOG аргыг ашиглан онцлог шинжийг олж дараах томьёогоор хамгийн өндөр оноотой байршлыг тогтоон илрүүлдэг. Зураг \ref{fig:dpm} -д зургийг пирамид загвар руу шилжүүлж илрүүлэлт хийх процессыг харуулж байна.
\begin{figure}[H]
    \centering
    \includegraphics[scale=0.85]{Figures/chapter1/dpm.PNG}
    \caption{DPM илрүүлэлт}
    \label{fig:dpm}
\end{figure}


%-------------------------------------------------------------------------------

\section{Объект илрүүлэлтэд ашиглаж буй сүүлийн үеийн алгоритмууд}
\paragraph{} Объект илрүүлэлтийн сүүлийн үеийн гол архитектур болох CNN дээр тулгуурлан хөгжүүлэгдсэн RCNN -н архитектурууд, YOLO болон SSD илрүүлэлтийн тухай 1.3 -н дэд бүлгүүдэд авч үзнэ.
\subsection{RCNN, Fast RCNN, Faster RCNN илрүүлэлт}
\par
\textbf{RCNN /Region-based Convolutional Neural Network/} нь 
бүсчлэлд суурилсан эргэлдсэн нейроны сүлжээ буюу компьютерын хараа болон дүрс боловсруулахад ашигладаг машин сургалтын загваруудын нэг. R-CNN-ийн анхны зорилго нь объект илрүүлэхэд зориулагдсан бөгөөд тэдгээрийн эргэн тойрон дахь хил хязгаарыг тодорхойлсон оролтын дүрс дэх объектуудыг илрүүлэх явдал юм \cite{introdRCNN}. R-CNN загварт өгөгдсөн оролтын зураг нь сонгомол хайлт хэмээх механизмаар дамждаг бөгөөд тухайн бүсчлэлийн талаарх мэдээллийг гаргаж авдаг. Сонирхсон бүсийг тэгш өнцөгтийн хилээр төлөөлж болно. Хувилбараас хамааран 2000 гаруй сонирхлын бүс байж болно. Энэхүү ашиг сонирхлын бүс нь CNN-ээр дамжуулж гаралтын шинж чанарыг бий болгодог. Дараа нь эдгээр гаралтын функцүүд нь SVM (Support Vector Machine/дэмжих вектор машин) ангилагчаар дамждаг бөгөөд тухайн объектыг сонирхож буй бүсэд ангилдаг. Зураг \ref{fig:wholeprocessRCNN} -д RCNN илрүүлэлтийн процессыг харуулж байна.
\begin{figure}[H]
    \centering
    \includegraphics[scale=0.55]{Figures/chapter1/wholeprocessRCNN.PNG}
    \caption{RCNN -н процесс}
    \label{fig:wholeprocessRCNN}
\end{figure}
\par
Зурган дээр өөр өөр хэмжээтэй гүйдэг шүүлтүүрийг ашиглан зургаас объектыг гаргаж авах нь бүрэн хайлтын арга гэж нэрлэгддэг. Сонгомол хайлтын алгоритм нь бүрэн хайлтыг дангаар нь ашиглахын оронд зураг дээр үзүүлсэн өнгөний сегментчилэлтэй ажилладаг. Сонгомол хайлт нь объектод өөр өөр өнгө өгөх замаар объектыг дүрсээс салгах арга гэж  хэлж болно.
Энэ алгоритм нь олон жижиг цонх эсвэл шүүлтүүр хийхээс эхэлж, сонирхлын бүсчлэлийг тодорхойлохын тулд greedy алгоритмыг ашигладаг. Дараа нь бүсчлэлд ижил төстэй өнгийг олж, тэдгээрийг нэгтгэдэг. Бүсчлэлийн ижил төстэй байдлыг дараах байдлаар тооцоолж болно. S(a,b)=Бүтэц(a,b)+Хэмжээ(a,b) Бүтэц(a,b) нь бүс нутгийн хоорондох харагдах байдал ба Хэмжээ(a,b) ижил төстэй байдал юм. Зураг \ref{fig:selectivealgorithm} -д сонгомол хайлтын арга болох өнгөний сегментчлэлийг ашигласан зургийг харуулж байна.
\begin{figure}[H]
    \centering
    \includegraphics[scale=0.5]{Figures/chapter1/selectivealgorithm.PNG}
    \caption{Сонгомол хайлтын алгоритмын дүрслэл}
    \label{fig:selectivealgorithm}
\end{figure}
\par Энэ үйлдлийн дараа эдгээр үйлдлүүд хийгдэнэ.
\begin{description}
  \item [1.Хувиргах] Бүсчлэлийг сонгосны дараа бүсчлэлүүдийг агуулсан зураг нь CNN-ээр дамждаг бөгөөд CNN загвар нь тухайн бүсээс объектуудыг гаргаж авдаг. Зургийн хэмжээг CNN-ийн багтаамжийн дагуу тохируулах ёстой тул дүрсийг өөрчлөхөд хэсэг хугацаа эсвэл ихэнх цаг хугацаа шаардагдана. Үндсэн R-CNN-д бүсчлэлийг 227 x 227 x 3 хэмжээтэй зураг болгон хуваадаг. Зураг \ref{fig:wrapping} -д хувиргах үйлдлийг харуулж байна.
  \begin{figure}[H]
    \centering
    \includegraphics[scale=0.4]{Figures/chapter1/wrapping.PNG}
    \caption{Хувиргах үйлдэл}
    \label{fig:wrapping}
\end{figure}
\item [2.CNN -р объектуудыг өргөжүүлэх] Объектыг задлахын тулд CNN-ийн эргэлдсэн оролтыг боловсруулна. Зураг \ref{fig:extractobjwithcnn} -с CNN -р объектуудыг өргөжүүлэх процессыг харж болно.
\begin{figure}[H]
    \centering
    \includegraphics[scale=0.65]{Figures/chapter1/extractobjwithcnn.PNG}
    \caption{CNN -р объектуудыг өргөжүүлэх процесс}
    \label{fig:extractobjwithcnn}
\end{figure}
  \item [3.Ангилах] Үндсэн R-CNN нь өөр өөр объектуудыг ангилалдаа тусгаарлах SVM ангилагчаас бүрддэг. Зураг \ref{fig:classification} -д ангилах процесс хэрхэн явагддагийг харуулж байна.
  \begin{figure}[H]
    \centering
    \includegraphics[scale=0.5]{Figures/chapter1/classification.PNG}
    \caption{Ангилах процесс}
    \label{fig:classification}
\end{figure}
\end{description}

\par
\textbf{Fast RCNN /Region-based Convolutional Neural Network/} нь 
RCNN -ээс илүү хурдан ажилладаг архитектур. Учир нь RCNN шиг бүсчлэл болгонд CNN -г хэрэгжүүлэхийн оронд эхлээд оролтын зурагтаа CNN -г бүхэлд нь хэрэгжүүлж дараа нь "сонирхлын бүсчлэл" нэртэй давхаргыг үүсгэн тухайн давхаргаасаа бүсчлэлүүдээ гаргаж авна, дараа нь бүрэн холбогдсон давхаргаар дамжуулж ажиллана. Объектыг хүрээлсэн хүрээг тодорхойлохдоо шугаман регресс хийдэг болсон нь өмнөх RCNN илүү боловсронгуй болгосон. Зураг \ref{fig:fastRCNN} -д Fast RCNN -н процессыг дүрсэлсэн.
\begin{figure}[H]
    \centering
    \includegraphics[scale=0.8]{Figures/chapter1/fastRCNN.PNG}
    \caption{Fast RCNN -н процесс}
    \label{fig:fastRCNN}
\end{figure}
\par
\textbf{Fastest RCNN /Region-based Convolutional Neural Network/} нь 
хамгийн хурдан ажилладаг RCNN архитектур. Илүү хурдан R-CNN нь бүсчлэлийн багцыг үүсгэхийн тулд бүсчлэлийн саналын аргыг ашигладаг \cite{introdRCNN}. Илүү хурдан R-CNN нь бүс нутгийн саналыг авах нэмэлт CNN-тэй бөгөөд үүнийг бүсчлэлийн саналын сүлжээ гэж нэрлэдэг. Сургалтын үед бүсчлэлийн саналын сүлжээ нь feature map -г оролт, гаралт болгон бүсчлэлийн саналыг авдаг. Эдгээр саналууд нь цаашдын процедурын хувьд сонирхлын бүсчлэлийг нэгтгэх давхаргад очдог. Зураг \ref{fig:fastestRCNN} -д Fastest RCNN -н процессыг дүрсэлсэн.
\begin{figure}[H]
    \centering
    \includegraphics[scale=0.75]{Figures/chapter1/fastestRCNN.PNG}
    \caption{Fastest RCNN -н процесс}
    \label{fig:fastestRCNN}
\end{figure}
\begin{longtable}{| p{3cm} | p{4cm} | p{4cm}| p{4cm}|}
\caption{RCNN архитектуруудын харьцуулалт}
\label{tab:husnegt1}\\
\hline
{\bfseries } & {\bf RCNN} & {\bf Fast RCNN} & {\bfseries Fastest RCNN} \\ \hline
{\bf Ашиглагддаг арга}& Сонгомол хайлт OS & Сонгомол хайлт & Бүсчлэлийн саналын арга\\ \hline
{\bf {Таамаглах хугацаа}} & 40-50сек & 2сек & 0.2сек\\ \hline
{\bf Тооцоолол}& Тооцооллын хугацаа өндөр &   Тооцооллын хугацаа өндөр &  Тооцооллын хугацаа бага \\ \hline

\end{longtable}
\subsection{SSD /Single Shot Detector/ илрүүлэлт }
\par 
SSD нь үндсэн загвар ба SSD толгой гэсэн хоёр бүрэлдэхүүн хэсэгтэй. Үндсэн загвар нь ихэвчлэн урьдчилан бэлтгэгдсэн дүрс ангилах сүлжээ юм \cite{introdSSD}. Эцсийн бүрэн холбогдсон ангиллын давхаргыг устгасан ResNET, MobileNet шиг. Үндсэн загварын сүлжээнүүдээс feature map -ууд гарч ирдэг. SSD толгой нь зөвхөн нэг буюу хэд хэдэн эргэлтийн давхарга бөгөөд гаралт нь эцсийн давхаргын үр дүнгийн орон зайн байршил дахь объектуудын хязгаарлагдмал хайрцаг, ангиллаар тайлбарлагддаг. SSD бол зөвхөн нэг алхмаар зургаас feature map, таамаглалыг гарган авч, итгэлийн оноог тооцоолон, match хийгээд хамгийн өндөр оноотой ангийг сонгон тухайн объектод харгалзах хэмжээгээр хүрээ үүсгэн буцаадаг архитектур юм.Зураг \ref{fig:ssdprocess} -д дүрслэгдсэн эхний хэдэн давхарга (цагаан хайрцаг) нь үндсэн давхарга, сүүлийн хэдэн давхарга (цэнхэр хайрцаг) нь SSD толгойг илэрхийлнэ.
\begin{figure}[H]
    \centering
    \includegraphics[scale=0.35]{Figures/chapter1/ssd.png}
    \caption{SSD -н процесс}
    \label{fig:ssdprocess}
\end{figure}
\subsection{YOLO /You Only Look Once/ илрүүлэлт }
\par YOLO гэдэг нь зураг дээрх янз бүрийн объектыг (бодит цаг хугацаанд) илрүүлж, таних алгоритм юм. YOLO дахь объект илрүүлэлт нь регрессийн бодлого хэлбэрээр хийгддэг бөгөөд илрүүлсэн зургийн ангиллын магадлалыг өгдөг. YOLO алгоритм нь объектуудыг бодит цаг хугацаанд нь илрүүлэхийн тулд convolutional neural network (CNN) ашигладаг \cite{introdYOLO}. Нэрээс нь харахад алгоритм нь объектыг илрүүлэхийн тулд нейроны сүлжээг зөвхөн нэг удаа л ашигладаг. Энэ нь бүхэл бүтэн зураг дээрх таамаглалыг нэг алгоритмын дагуу хийдэг гэсэн үг юм. 
YOLO алгоритм нь дараах гурван техникийг ашиглан ажилладаг.
\begin{itemize}
    \item Үлдэгдэл блокууд
    \item Объектыг хүрээлэх хүрээний регресс
    \item Intersection Over Union (IOU)
\end{itemize}
Нэгдүгээрт, зургийг янз бүрийн сүлжээнд хуваана. Сүлжээ бүр S x S хэмжээтэй байна. Зураг \ref{fig:residual} дээр ижил хэмжээтэй олон тор нүд байна. Зураг дээрх сүлжээний нүд бүр тэдгээрийн дотор гарч буй объектуудыг илрүүлэх болно. Жишээлбэл, хэрэв тодорхой сүлжээний нүдэнд объектын төв гарч ирвэл энэ нүд үүнийг илрүүлэх үүрэгтэй. 
\begin{figure}[H]
    \centering
    \includegraphics[scale=0.34]{Figures/chapter1/residual.png}
    \caption{Зургийг блокуудад хуваах}
    \label{fig:residual}
\end{figure}
\par Объектыг хүрээлэх хүрээний регресс :
Объектыг хүрээлэх хүрээ нь зураг дээрх объектыг тодотгох тойм юм.
Зурган дээрх объектыг хүрээлэх хүрээ бүр дараах шинж чанаруудаас бүрдэнэ.Өргөн (bw), өндөр (bh), анги (жишээлбэл, хүн, машин, гэрлэн дохио гэх мэт)- Үүнийг c үсгээр илэрхийлнэ.
Хязгаарлах хайрцгийн төв (bx, by). Зураг \ref{fig:boundingbox} -д объектыг хүрээлэх хүрээ болон хүрээний регрессийг харуулж байна.
\begin{figure}[H]
    \centering
    \includegraphics[scale=1]{Figures/chapter1/boundingbox.PNG}
    \caption{Объектыг хүрээлэх хүрээ}
    \label{fig:boundingbox}
\end{figure}
\par YOLO нь объектын өндөр, өргөн, төв, ангиллыг таамаглахын тулд нэг хязгаарлах хайрцгийн регрессийг ашигладаг. Дээрх зурган дээрх объектын хязгаарлах хайрцагт гарч ирэх магадлалыг илэрхийлнэ.

Intersection over union (IOU) нь объект илрүүлэх үзэгдэл бөгөөд хайрцгууд хэрхэн давхцаж байгааг тодорхойлдог. YOLO нь объектуудыг төгс хүрээлэх гаралтын хүрээг хангахын тулд IOU-г ашигладаг \cite{introdYOLO}.  Сүлжээний нүд бүр нь объектыг хүрээлэх хүрээ болон тэдгээрийн итгэлийн оноог урьдчилан таамаглах үүрэгтэй. Урьдчилан таамагласан хязгаарын хайрцаг нь бодит хайрцагтай ижил байвал IOU нь 1-тэй тэнцүү байна. Энэ механизм нь жинхэнэ хайрцагтай тэнцүү биш хязгаарлах хайрцгийг арилгадаг. Зураг \ref{fig:intersection} нь IOU хэрхэн ажилладаг тухай энгийн жишээг харуулж байна.
\begin{figure}[H]
    \centering
    \includegraphics[scale=0.4]{Figures/chapter1/intersection.jpeg}
    \caption{Объектуудыг хүрээлсэн хүрээ огтлолцох}
    \label{fig:intersection}
\end{figure}
\par











\section{Ижил төстэй системийн судалгаа}
\paragraph{} “Upload” үйлдэл хийж оруулсан зургаас зөвхөн объект илрүүлэлт хийж чаддаг “Iashin” вэбсайт болон хараагүй хүмүүсийн суралцдаг Перкинс сургууль болон Техасын харааны бэрхшээлтэй хүмүүсийн сургууль (TSBVI) хоёрын хамтын вэбсайт болох Paths to Literacy -н судалгаагаар харааны бэрхшээлтэй иргэдэд тусалж чадах хамгийн шилдэг аппликейшн гэж зарлагдсан “TapTapSee” гар утасны аппликейшныг ижил төстэй системээр сонгон судална.
\par \textbf{Iashin.ai вебсайт :}  2019 онд анх интернэт сүлжээнд тавигдсан Владимир Иашин гэх судлаач, видео контентыг ойлгох нейроны сүлжээг зохион бүтээх чиглэлээр мэргэшсэн профессорын судалгааны ажилдаа зориулан хөгжүүлсэн үнэ төлбөргүй веб сайт юм \cite{iashin}. Вэбсайтын хувьд back-end болон front-end гэсэн 2 хэсгээс бүрдэнэ. Front-end нь javascript ашиглан веб -д хуулагдсан зургийг уншиж авч дараа нь илгээх үүрэгтэй. Харин back-end хэсэг нь веб аппликейшн бий болгоход зориулагдсан Flask фреймворкыг ашигладаг бол өгөгдсөн зургаас объект илрүүлэлтийг хийхдээ YOLOv3 архитектур, COCO dataset,  PyTorch машин сургалтын фреймворкыг ашигладаг. Өгөгдсөн зургаас 80 төрлийн зүйлийг олж чадна \cite{iashin}. Зураг \ref{fig: iashinweb1} -д Iashin.ai вэбсайтын нүүр хуудсыг харуулж байна.
\begin{figure}[H]
    \centering
    \includegraphics[scale=0.85]{Figures/chapter1/iashinweb1.PNG}
    \caption{Iashin.ai веб сайтын нүүр хуудас}
    \label{fig: iashinweb1}
\end{figure}
\par \noindent Зураг \ref{fig: iashinweb} -д  iashin.ai сайтын “upload” хийж оруулсан зурагт хийсэн объект илрүүлэлтийн үр дүнг харуулж байна. Тус сайт нь объектыг хүрээлсэн хүрээ болон түүний ямар ангилалд хамгийн ихдээ хэд хувиар ижил байгааг зурагт дүрслэн харуулдаг.
\begin{figure}[H]
    \centering
    \includegraphics[scale=0.35]{Figures/chapter1/iashinweb.PNG}
    \caption{Iashin.ai веб сайт зурганд объект илрүүлэлт хийсэн нь }
    \label{fig: iashinweb}
\end{figure}

\par \textbf{TapTapSee мобайл аппликейшн :} нь хараагүй болон харааны бэрхшээлтэй хэрэглэгчдэд тусгайлан зориулсан гар утасны үнэгүй программ юм. Тус аппликейшныг 2012онд CloudSight компани анхны мобайл аппликейшнаа болгон гаргаж байсан \cite{cloudsightinc}. Гар утасны Android болон IOS үйлдлийн системтэй хэрэглэгчид ашиглах боломжтой. TapTapSee нь төхөөрөмжийн камераар аливаа зүйлийн зураг, видеог авч, VoiceOver функцийг ашиглан илэрсэн объектуудыг хэрэглэгчид чанга дуугаар уншиж өгдөг. Объект илрүүлэлтдээ CloudSight -аас хөгжүүлсэн CloudSight API -г ашигладаг бөгөөд энэ нь хүний тархийг дуурайдаг, цаг хугацааны явцад алдаанаасаа "суралцдаг" гүн гүнзгий суралцах технологи, одоогоор энэ API нэг тэрбум гаруй зургийг зөв таньсан \cite{taptapsee}. CloudSight API нь Large Language Model буюу их хэмжээний шошгогүй өгөгдөл дээр сургагддаг self-supervised сургалтыг ашигласан нэг төрлийн нейроны сүлжээний загвар \cite{cloudsightapi}. Энэ загвар нь ерөнхийдөө Transformer neural network буюу хувиргагч нейроны сүлжээг авч ашигласан байдаг.
\begin{figure}[H]
    \centering
    \includegraphics[scale=1]{Figures/chapter1/taptapsee.PNG}
    \caption{TapTapSee аппликейшны лого}
    \label{fig: taptapsee}
\end{figure}
\par \noindent Дэлгэцийн баруун талд хоёр товшиж зураг авах эсвэл дэлгэцийн зүүн талд давхар товшоод видео хийх боломжтой. TapTapSee нь секундийнн дотор аль ч өнцгөөс хоёр буюу гурван хэмжээст объектыг үнэн зөв тодорхойлох боломжтой. Төхөөрөмжийн доод хэсэгт илэрсэн объектууд текст хэлбэрээр гарч ирэх бол дараа нь төхөөрөмжийн VoiceOver функц тус текст мэдээллийг хэрэглэгчид чангаар хэлж өгдөг. Зураг \ref{fig: taptapsee2} -д TapTapSee аппликейшн гар утсан дээр ажиллаж байгааг харуулж байна.
\begin{figure}[H]
    \centering
    \includegraphics[scale=0.8]{Figures/chapter1/taptapsee2.PNG}
    \caption{TapTapSee аппликейшн гар утсан дээр ажиллах байдал}
    \label{fig: taptapsee2}
\end{figure}

\begin{longtable}{| p{7cm} | p{4cm} | p{4cm} |}
\caption{Судлагдсан системүүдийн харьцуулалт}
\label{tab:husnegt2}\\
\hline
{\bfseries  Системүүд } & {\bf Vladimir Iashin веб} & {\bf TapTapSee мобайл аппликейшн} \\ \hline
{\bf Платформын төрөл}& Веб аппликейшн& Мобайл аппликейшн \\ \hline
{\bf {Объектын байршил харуулах}} & Тийм & Үгүй\\ \hline
{\bf {Илэрсэн объектуудын мэдээллийг текст руу хөрвүүлэх}} & Үгүй & Тийм\\ \hline
{\bf {Илэрсэн объектуудын мэдээллийг аудио руу хөрвүүлэх}} & Үгүй & Тийм\\ \hline
{\bf {Видеоноос объект илрүүлэлт хийх}} & Үгүй & Тийм\\ \hline
{\bf {Илрүүлж чадах объектын нэр төрлийн тоо}} & 80 &  80<\\ \hline
{\bf {Төлбөртэй эсэх}} & Үгүй & Үгүй\\ \hline
\end{longtable}
\section{Хөгжүүлэлтэд ашиглах алгоритм}
\paragraph{} Энэ бүлэг нь системийн хөгжүүлэлтийн гол хэсэг болох объект илрүүлэлтэд ашиглах алгоритмыг тайлбарлана. Дэлгэрэнгүй тайлбар болон судалгааг 2.2.1 -р дэд бүлэгт авч үзнэ.
\subsection{Объект илрүүлэлтэд ашиглах алгоритм }
\paragraph{}
\textbf{MobileNet – SSD V3} буюу Microsoft -н COCO dataset дээр сургагдсан моделийг системийн back-end хөгжүүлэлтийн объект илрүүлэлтэд ашиглана. 
\textbf{MobileNet} бол бага хэмжээний CPU ашиглах боломжийг олгодог нэг төрлийн жижиг хэмжээний, бага чадалтай CNN/Convolutional Neural Network/ aрхитектур. Анх 2017онд Google компанийн инженерүүд “Мобайл вишн аппликейшн -д илүү үр нөлөөтэй эргэлдсэн нейроны сүлжээ” /Efficient Convolutional Neural Network for Mobile Vision Application/ гэх судалгааны ажил дээрээ танилцуулж байсан \cite{mobilenet}. 
\textbf{SSD /Single Shot Detector/}нь нэг алхамт объект илрүүлэгч архитектур. 2016 онд Google компанийн судалгааны багийнхан бодит цагийн горимд хоцрогдолгүйгээр, цөөн алхмаар ажиллаж чадах объект илрүүлэх архитектур гаргасан нь SSD байсан. SSD -н процесс нь 
\begin{enumerate}
    \item “Feature map” – уудыг өргөтгөх 
    \item Объектыг илрүүлэх     гэсэн үндсэн хоёр хэсэгт хуваагддаг. 
\end{enumerate}
SSD нь “base network” буюу “Feature map” – уудыг өргөтгөх процессоо тусдаа ажиллуулах боломжтой байдаг учраас эхний процессыг MobileNet архитектурыг ашиглан ажиллуулж, дараагийн процессыг SSD гүйцэтгэнэ.
MobileNet –г Convolution үйлдэл хийлгэж, feature map -уудыг өргөтгөхийн тулд ашигладаг \cite{ssd}. Зураг \ref{fig: mobSSDarc} нь MobileNet - SSD V3 -н архитектурыг харуулж байна.
\begin{figure}[H]
    \centering
    \includegraphics[scale=1]{Figures/chapter1/mobilenetSSD-architecture.PNG}
    \caption{MobileNet - SSD V3 архитектур}
    \label{fig: mobSSDarc}
\end{figure}
\paragraph{} \textbf{Convolution үйлдэл :} Тус давхаргад зургийг матриц хэлбэрээр хүлээн авна. Ямар ч зургийг цэгэн утгуудын матрицаар илэрхийлж болдог. Өнгөт зургийг дэлгэцэд харуулахад улаан, ногоон, цэнхэр (Red, Green, Blue = RGB) өнгөний хослолоор шийддэг. Үүнийг, өнгө бүр нь нэг давхарга болж, дээр дээрээс нь давхарласан гурван давхарга гэж төсөөлж болох бөгөөд давхарга бүр дээрх цэгийн утга 0–255 хооронд тоон утга авна. Харин хар цагаан зураг бол нэг сувагтай. Нэг сувагтай гэдэг нь өнгөний нэг л давхарга байна гэсэн үг. Давхарга дээрх цэг бүрийн утга мөн л 0–255 хооронд байна. 0 (тэг) бол хар, 255 бол цагаан өнгө болно. /Энгийнээр, өнгөт зураг бол 3 давхаргаар дүрслэгдэнэ, хар зураг бол нэг давхаргаар илэрхийлэгдэнэ. Давхарга(матриц) бүрийн утгууд 0-255 хооронд байна./  \newline
 Цонх матриц буюу Filter/мөн kernel эсвэл feature detector ч гэж нэрлэдэг/ - ийг үүсгэн түүнийгээ оролтын зурган матриц дээгүүр тодорхой алхмаар гүйлгэж тодорхой feature - үүдийг агуулсан шинэ матриц үүсгэдэг. Оролтын матриц болон цонх матрицын үржвэр, нийлбэрээс үүссэн гаралтын энэ матрицыг ‘Convolved Feature’ эсвэл ‘Activation Map’ эсвэл ‘Feature Map‘ гэж нэрлэнэ. Зураг \ref{fig: kernelling} -д зураг дээгүүр 3х3 хэмжээтэй Filter гүйлгэснээр ‘Feature Map‘ хэрхэн үүсэж байгааг харуулж байна.
 \begin{figure}[H]
    \centering
    \includegraphics[scale=1.3]{Figures/chapter1/kernelling.PNG}
    \caption{Filter - хийж буй процесс}
    \label{fig: kernelling}
\end{figure}
\paragraph{} Шүүлтүүрийн тоо, түүний хэмжээ, сүлжээ (CNN) -ний архитектур бүтэц зэргийг сургалтаас өмнө бид тодорхойлж өгнө.
Хэдий чинээ олон фильтертэй байна төдийн чинээ олон шинж тодорхойлогдоно. Ингэснээр тухайн зургаас суралцах зүйл ихсэж, өөр шинэ зургийг танихад илүү хэрэгтэй болно.Feature Map (Convolved Feature) -ын хэмжээ дараах гурван параметрээр тодорхойлогдоно.
\begin{enumerate}
    \item \textbf{Depth буюу Гүн}: Энэ параметр нь convolution үйлдэлд хэдэн фильтер байхыг заана
    \item \textbf{Stride буюу Алхам}: Энэ бол оролтын матриц дээгүүр шүүлтүүр матрицыг нэг удаад шилжүүлэх цэгийн тоо юм. Хэрэв алхам 1 гэвэл шүүлтүүр 1 цэгээр шилжинэ. Хэрэв алхам 2 гэвэл шүүлтүүр 2 цэгээр шилжинэ. Алхмын тоо том байвал feature map -ын хэмжээ жижиг болно.
    \item \textbf{Zero-padding буюу Хоосон зай}: Оролтын матрицын гадна хүрээгээр хоосон зай (0=тэг утга) оруулах нь зарим тохиолдолд тохиромжтой байдаг. Ингэсэн үед шүүлтүүр нь оролтын матрицын ирмэгийн цэгт ч хүрч чадна. Мөн feature map -ын хэмжээг удирдах боломж олгоно. Хоосон зай оруулсан бол wide convolution, хоосон зай оруулаагүй бол narrow convolution гэж нэрлэдэг.
\end{enumerate}

\par \noindent Зураг \ref{fig: CNN filters} -с нэг л зураг өгөгдсөн байхад ялгаатай фильтерүүд ялгаатай шинжийг (ирмэг, зураас, тойрог, өнгө гэх мэт) тодруулж байгааг харж болно.
\begin{figure}[H]
    \centering
    \includegraphics[scale=1]{Figures/chapter1/cnn-filters.PNG}
    \caption{Filter - н төрлүүд}
    \label{fig: CNN filters}
\end{figure}
\par \noindent Convolution -ы layer бүр дээр SSD нь объект илрүүлэлт хийнэ. Ингэхдээ convolution үйлдлээс гарч ирсэн layer -н channel болгоныг 38х38, 19х19,10х10,6х6 гэх мэт grid болгож grid –н  нэг нүднээс 4-6 таамаглал гарган авдаг. Том хэмжээний объект нь жижиг хэмжээтэй grid -г ашигласан үед, жижиг хэмжээний объект нь том хэмжээтэй grid -г ашигласан үед олддог. Зураг \ref{fig: getPredictions} -д өөр өөр хэмжээтэй grid -с таамаглал хэрхэн гарч байгааг харуулж байна.
\begin{figure}[H]
    \centering
    \includegraphics[scale=1]{Figures/chapter1/featureMaps.PNG}
    \caption{Таамаглалыг сонгон авах}
    \label{fig: getPredictions}
\end{figure}

\par SSD нь таамагласан объектыг хүрээлж буй хүрээний өндөр, өргөнөө (1,2,3,0.5,0.3) гэсэн харьцаагаар бодон таамаглал бүртээ default хэмжээтэй хүрээ өгдөг. SSD -н талаарх энэхүү ойлголтын илүү ойлгомжтой дүрслэлийг Зураг \ref{fig: predictionsOnNode} -с харж болно.
\begin{figure}[H]
    \centering
    \includegraphics[scale=0.6]{Figures/chapter1/predictions.PNG}
    \caption{Нэг цэг дээрх таамаглалууд}
    \label{fig: predictionsOnNode}
\end{figure}
\par Таамаглал бүрд өгч буй объектыг хүрээлсэн хүрээний урт, өргөнийг тооцоолохдоо Зураг \ref{fig: f1} -д үзүүлсэн томьёогоор олно.
\begin{figure}[H]
    \centering
    \includegraphics[scale=1]{Figures/chapter1/f1.PNG}
    \caption{Объектыг хүрээлсэн хүрээний урт, өргөнийг тооцоолох томьёо}
    \label{fig: f1}
\end{figure} 


\par \noindent Зураг \ref{fig: multibox} -д үзүүлсний дагуу таамаглал бүрд multibox байна. Multibox нь дотроо тухайн таамаглалыг хүрээлж байгаа хүрээний х,у байршил, өндөр, өргөний хэмжээ болон таамагласан класс болгонд харгалзах итгэлийн оноо/confidence score/ -г агуулсан байна. SSD нь таамаглал бүрээ авч match хийнэ. Нэгтэй тэнцүү эсвэл нэгтэй хамгийн ойр итгэлийн оноотой классыг сонгон түүнийгээ объектын илрүүлэлт болгон авна.
\begin{figure}[H]
    \centering
    \includegraphics[scale=0.75]{Figures/chapter1/ssd-prediction.PNG}
    \caption{Таамаглалын multiBox }
    \label{fig: multibox}
\end{figure}
\par \textbf{MobileNet - SSD V3 моделийн илрүүлэлтийн нарийвчлал} : Малайз улсын “MARA Технологийн Их сургууль” -ийн “Компьютер Математикийн тэнхим”- ийн MobileNet SSD болон YOLO V3 илрүүлэлтийн архитектуруудыг  харьцуулсан судалгаа нь MobileNet SSD -н илрүүлэлтийн нарийвчлалын талаар дараах үр дүнг харуулсан. 
\begin{itemize}
    \item MobileNet SSD архитектурыг ашигласан сургалтыг явуулсан техник хангамжийн үзүүлэлт :  Intel Core i5-7300HQ 4 цөмт, хамгийн дээд давтамж нь 3.5 GHz хүрч чадах CPU, 8GB RAM, 4GB VRAM -тай NVIDIA -н GTX 1050 GPU. 
    \item Сургалтад ашигласан өгөгдөл : MS COCO dataset -н 2.5 сая шошгыг агуулсан 328000 зураг \cite{accuracy-mobilenet}
\end{itemize}
Сургалт нь нийт 5 цаг орчим үргэлжилж 128000 алхамт ажлыг гүйцэтгэсэн. Объект илрүүлэлтийн архитектуруудын илрүүлэлтийн нарийвчлалыг хэмждэг нэгж mAP буюу дундаж нарийвчлалаар үр дүнг хэмжсэн. \cite{accuracy-mobilenet} mAP нь нарийвчлалыг 0-1 гэсэн утгын хооронд хэмждэг. Зураг \ref{fig: map-mobilenet} -д үзүүлсний дагуу сургалтын эхэн үед илрүүлэлтийн нарийвчлал 0.4mAP байсан бол 4 дэх цагаас эхлэн 0.53mAP орчим болж тогтворжиж байна. 
\begin{figure}[H]
    \centering
    \includegraphics[scale=0.8]{Figures/chapter1/map-mobilenet.PNG}
    \caption{MobileNet SSD архитектурын дундаж нарийвчлал}
    \label{fig: map-mobilenet}
\end{figure}
\par Хүснэгт \ref{tab:husnegt-map-mobilenet} -д судалгааны үр дүнд гарсан MobileNet SSD архитектурын илрүүлэлтийн нарийвчлалыг харуулж байна.
\begin{longtable}{| p{9cm} | p{4cm} |}
\caption{MobileNet SSD архитектурын үзүүлэлтүүд}
\label{tab:husnegt-map-mobilenet}\\
\hline
{\bfseries \centering Үр дүн} & {\bf MobileNet SSD V3} \\ \hline
{mAP}& \centering 53.29\%  \ \hline
{mAP /Дундаж хэмжээтэй Объект/} & \centering 58.33\%  \ \hline
{mAP /Жижиг хэмжээтэй Объект/} & \centering 53.34\%  \ \hline
{mAP(0.5IOU)} & \centering 97.8\%  \ \hline
{mAP(0.75IOU)} & \centering 51.28\%  \ \hline
{Илрүүлэлтийн хурд} & \centering 135.5мс\ \hline
\end{longtable}
Объект илрүүлэлтийн модел болгоны объект илрүүлэлтийн нарийвчлал нь ашиглаж байгаа архитектур, өгөгдлийн тоо, сургалтын хугацаа болон сургалт явуулж байгаа техник хангамжаас шалтгаалан бага зэргийн зөрүүтэй байх боломжтой.  




%-------------------------------------------------------------------------------

\section{Технологийн судалгаа}
\paragraph{} Системийн back-end болон front-end хөгжүүлэлтэд ашиглах технологиуд болон хөгжүүлэлтийг хэрэгжүүлэхэд шаардлагатай хөгжүүлэлтийн орчны тайлбар судалгаа зэргийг агуулсан.
\subsection{Back-End хөгжүүлэлтэд ашиглах технологиуд}
\subsubsection{OpenCV зургийн сан}
\par
OpenCV/Open Computer Vision/ нь 1999 онд Intel -н CPU ихээр ашигладаг программуудад зориулсан, С++ хэл дээр хөгжүүлэгдсэн сан. Одоо OpenCV нь Apache 2.0 лизенцийн дагуу нээлттэй эхийн компьютер вишн, машин сургалтад ашиглагддаг 2500+ алгоритмыг агуулсан сан юм. OpenCV нь Python, MATLAB, Java хэлийг дэмждэг, Linux, Windows, macOS, OpenBSD, NetBSD, FreeBSD үйлдлийн систем дээр ажиллаж чадна. Хэрвээ OpenCV нь системээс Intel -н Integrated Performance Primitivies/Нэгдсэн гүйцэтгэлийн командууд/-г олж чадвал оновчтой горимуудыг ашиглан өөрийгөө хурдасгаж чадна \cite{OpenCV}. 
\begin{figure}[H]
    \centering
    \includegraphics[scale=0.8]{Figures/chapter1/opencv.png}
    \caption{ОpenCV лого}
    \label{fig: opencv logo}
\end{figure}
\par \noindent Царай таних, дохио зангаа таних, видео дээрх хөдөлгөөнийг бүртгэн авах, AR/Aug- mented Reality/ зэрэг компьютер вишн -н олон төрөлд ашиглах боломжтой сангуудыг агуулсан. Мөн машин сургалтын "boosting" алгоритм, Бейсийн тодорхойлогч алгоритм, шийдвэрийн модны сургалтын алгоритм, "k-nearest neighbour" алгоритм гэх мэт олон алгоритмуудыг хэрэгжүүлдэг. Зургаас объект илрүүлэлтийг хийхийн тулд OpenCV -гийн  DNN пакежийн Detec- tionModel классыг ашиглана.  Энэ класс нь объект илрүүлэлтэнд зориулсан өндөр түвшний API -г илэрхийлдэг, сургагдсан моделийн weight болон configuration файлыг ашиглан зургаас объект илрүүлэлт хийх сүлжээг бий болгодог. Ингэснээр аль хэдийн сургагдсан моделийн weight болон давхаргуудаар сүлжээ үүсгэн, тэр сүлжээг ашиглан зургаас объект илрүүлэлтийг хийнэ. Энэ класс нь байгуулагч функцэдээ 2 аргумент авдаг. Эхнийх нь сургагдсан моделийн weight -г агуулсан файл, сүүлийнх нь configuration файл байна. DetectionModel класс нь SSD, Faster R-CNN, YOLO архитектураар хөгжүүлэгдсэн моделийг дэмждэг. 
\subsubsection{FAST API}
\par
FAST API нь 2018 онд анхны хувилбараа гаргаж байсан, MIT License -тэй Python3.7+ дээр API/Application Programming Interface/ үүсгэхэд зориулагдсан орчин үеийн хурдан, өндөр гүйцэтгэлтэй веб фреймворк юм.
\par Гол онцлогууд нь :
\begin{itemize}
  \item \textbf{Хурдан: }NodeJS болон GO -тэй ижил өндөр гүйцэтгэлтэй, хамгийн хурдан Python веб фреймворкуудын нэг.
  \item \textbf{Хурдан кодчлол: }Фреймворкуудын онцлогуудыг ашигласнаар хөгжүүлэлтийн хурдыг 200\%-300\% -р нэмэгдүүлнэ.
  \item \textbf{Цөөн алдаа: }Хөгжүүлэгчээс үүдэлтэй алдааны 40 орчим хувийг бууруулна.
  \item \textbf{Хялбар: }Ашиглах болон сурахад хялбар байхаар бүтээгдсэн. Документ унших цагийг багасгасан.
  \item \textbf{Найдвартай: }Автомат интерактив баримт бичгийг бий болгодог.
\end{itemize}
\begin{figure}[H]
    \centering
    \includegraphics[scale=0.25]{Figures/chapter1/fastapi.png}
    \caption{FAST API лого}
    \label{fig: fastapi logo}
\end{figure}
\subsubsection{Python gTTS сан}
\paragraph{}
Python хэлний gTTS сан нь Google Translate -н текстийг яриа руу хөрвүүлдэг функцийг ашигладаг бөгөөд энэ нь Google Cloud Text-to-Speech -ээс ялгаатай. Энэ функцийг CLI түүл хэлбэрээр ч ашиглаж болно. Нийт 17 хэлийг таньж оролтын текстийг нь mp3 форматтай аудио болгож чадна.
\begin{figure}[H]
    \centering
    \includegraphics[scale=0.65]{Figures/chapter1/gtts.PNG}
    \caption{Google translate болон Python лого}
    \label{fig: texttospeech logo}
\end{figure}

\subsubsection{Chimege API}
\par
Chimege API нь Чимэгэ системс ХХК -ын гаргасан монгол хэл дээрх текстийг аудио руу, аудиог текст хэлбэр рүү хөрвүүлэх боломжийг олгодог. “Чимэгэ уншигч” endpoint бол хүссэн бичвэрийг хиймэл оюунаар боловсруулж, амьд мэт хоолойгоор уншдаг. Үнийн хувьд Chimege API -н нэг endpoint -г дуудахад 8 үнэтэй байдаг. Яриа руу хөрвүүлэх бичвэрийг мөн засаж хөрвүүлдэг. Жишээлбэл: 
\begin{itemize}
  \item Шаардлагагүй тэмдэгтүүдийг хасна.
  \item Түгээмэл товчилсон үгсийг задална. (МУ - Монгол Улс)
  \item Тоог задална. (123 - Зуун хорин гурав)
\end{itemize}
\begin{figure}[H]
    \centering
    \includegraphics[scale=0.2]{Figures/chapter1/chimege.png}
    \caption{Чимэгэ Системс лого}
    \label{fig: chimege logo}
\end{figure}
\subsubsection{MySQL өгөгдлийн сан удирдах систем}
\par
MySQL нь бүтэцлэгдсэн өгөгдлийн сангийн менежментийн систем. Анх Майкл Видениус зохион бүтээж байсан бөгөөд өөрийн охин My -ын нэр болон SQL хэлний нэрийг нийлүүлж нэрлэсэн бөгөөд анхны хувилбараа 1995 онд гаргасан. 2010 онд тус өгөгдлийн сангийн менежментийн системийг Oracle компани худалдаж авсан, одоо MySQL нь GNU General Public License-ийн нөхцөлийн дагуу үнэгүй, нээлттэй эхийн программ хангамж. MySQL нь C болон C++ хэл дээр бичигдсэн, AIX, BSDi, FreeBSD, HP-UX, ArcaOS, eComStation, IBM i, IRIX, Linux, macOS, Windows, NetBSD, Novell NetWare, OpenBSD, OpenSolaris, OS/2 Warp, QNX, зэрэг олон системийн платформ дээр ажилладаг.
\begin{figure}[H]
    \centering
    \includegraphics[scale=0.75]{Figures/chapter1/mysql.png}
    \caption{MySQL лого}
    \label{fig: mysql logo}
\end{figure}

\subsubsection{PyCharm IDE}
\par 
PyCharm нь Python, веб, өгөгдлийн шинжлэх ухааныг үр бүтээлтэй хөгжүүлэхэд тохиромжтой орчныг бүрдүүлэхийн тулд нягт нэгтгэсэн, Python хөгжүүлэгчдэд зориулсан өргөн хүрээний чухал хэрэгслээр хангагдсан Python-ийн нэгдсэн хөгжүүлэлтийн орчин (IDE) юм. Java болон Python хэл дээр JetBrains компаниар хөгжүүлэгдсэн, 2010 онд анхны хувилбараа гаргасан, Linux, Windows, macOS дээр ажиллах боломжтой. Нээлттэй эхийн Community болон төлбөртэй Professional хувилбаруудтай. Anaconda, IPython, Kite, Pylint, Pitest зэрэг машин сургалт, шинжлэх ухааны тооцоололд ашиглагддаг сангууд, плаг-ин, фреймворкуудыг агуулсан.
\begin{figure}[H]
    \centering
    \includegraphics[scale=0.9]{Figures/chapter1/pycharm.png}
    \caption{PyCharm IDE лого}
    \label{fig: pycharmide logo}
\end{figure}
\par PyCharm IDE -н давуу талууд :
\begin{itemize}
  \item Pandas, Scikit-learn зэрэг шинжлэх ухааны янз бүрийн номын санд хандах боломжтой.
  \item Линзний функцээр нарийвчилсан дүн шинжилгээ хийх боломжтой.
  \item Синтаксын алдааг олоход хялбар болгогч кодын алдаа засагчтай.
  \item Автоматаар бөглөх синтакс функц нь цаг хэмнэдэг
\end{itemize}
PyCharm IDE -н сул талууд :
\begin{itemize}
  \item Сурахад их цаг зарцуулдаг.
  \item Үндсэн скриптэд нэг их үнэ цэнэ нэмдэггүй.
  \item Олон нийтийн хувилбар нь бусад кодчиллын хэл рүү нэвтрэх эрхийг хязгаарладаг.
\end{itemize}
\subsection{Front-End хөгжүүлэлтэнд ашиглах технологиуд}
\subsubsection{React фреймворк}
React/ReactJS, React.js гэж бас хэлэгддэг/ нь веб аппликейшны User Interface/хэрэглэгчийн интерфейс/- г хөгжүүлэхэд зориулагдсан JavaScript фреймворк юм. 2011 онд Мета компанийн программ хангамжийн инженер Жордан Уолке JavaScript хөгжүүлэлтийн нарийн төвөгтэй байдлыг арилгахын тулд сomponents/бүрэлдэхүүн хэсгүүд/-д суурилсан, дахин ашиглах боломжтой React фреймворкыг хөгжүүлж, 2013 онд анхны хувилбараа гаргаж байсан.
\par
ReactJS -н давуу талууд :
\begin{itemize}
  \item Cуралцах болон хэрэглэхэд хялбар. Ойлгоход хялбар документ болон зааварчилгаатай. Хамгийн олон практик хичээлтэй фреймворк.
  \item JavaScript Extension -г ашиглаж динамик веб хөгжүүлэхэд хялбар болгосон.
  \item Virtual DOM буюу HTML, XHTML, XML -тай ажилладаг API -г ашигласнаар гүйцэтгэлийг нэмэгдүүлсэн.
  \item Дахин ашиглах боломжтой component/бүрэлдэхүүн хэсэг/ -г бий болгодог.
\end{itemize}
ReactJS -н сул талууд :
\begin{itemize}
  \item Байнгын шинэчлэл. Энэ нь хөгжүүлэгчдийн хувьд шинэчлэлтийг байнга судалж, хөгжүүлэлтдээ нэвтрүүлэх шаардлагатай байдаг учраас хөгжүүлэлтийг удаашруулах хандлагатай байдаг.
  \item Системийн хувьд зөвхөн хэрэглэгчийн интерфейсийн хэсгийг хамардаг учраас системийг бүхэлд нь хөгжүүлэхийн тулд бусад технологийг ч бас нэвтрүүлэх шаардлага гардаг.
\end{itemize}
\begin{figure}[H]
    \centering
    \includegraphics[scale=0.055]{Figures/chapter1/react.png}
    \caption{React лого}
    \label{fig: react logo}
\end{figure}
\subsubsection{Ant design}
\par
Ant Design нь Alibaba, Alipay, Huabei, MYbank-н толгой компани болох Ant Group-ийн боловсруулсан нээлттэй эхийн дизайны систем юм. 2017 онд анхны хувилбараа гаргаж байсан. Аnt Design -н бүрэлдэхүүн хэсгүүд нь React, Vue болон Angular гэх мэт front-end хөгжүүлэлтийн фреймворкуудыг дэмждэг. Тус системийн онцлог давуу талууд нь системийг хэрэглэх гол хүчин зүйл нь болж байдаг.
\par Ant Design -н онцлогууд:
\begin{itemize}
 \item \textbf{Арчилгаа сайтай: }Ant Design-ийн баг байнгын шинэчлэлтүүдээр дизайны системийг сайжруулахаар тасралтгүй ажилладаг.
  \item \textbf{Цогц дизайны сан: }Дизайны ямар ч асуудлыг шийдэж чадах бүрэлдэхүүн хэсэг, загвар болон тэмдэгт дүрсүүдийг агуулсан байдаг.Нэмж дурдахад элемент бүр нь ямар ч хувилбарт тохирох олон харагдацтай байдаг.
  \item \textbf{Интернационал: }Дэлхийн өнцөг булан бүрийн хэлийг дэмждэг бөгөөд хөгжүүлэгчид илүү ихийг нэмэх боломжтой.
  \item \textbf{Маягт: }Ямар ч төрлийн маягтыг бий болгож чадна.
  \item \textbf{Бэлэн загвар: }Өгөгдлийн үзүүлэлт, тайлан, график, админы интерфейс, чат, нэвтрэх хэсэг зэрэг олон төрлийн интерфейсүүдэд зориулсан 100+ бэлэн загваруудтай.
\end{itemize}
\begin{figure}[H]
    \centering
    \includegraphics[scale=1.2]{Figures/chapter1/antdesign.jpg}
    \caption{ Ant design лого}
    \label{fig: antdesign logo}
\end{figure}
\subsubsection{Visual Studio Code}
\par Visual Studio Code /VS Code гэж нэрлэдэг/ нь Microsoft, Electron Framework-тай хамтран хийсэн Windows, Linux болон macOS-д зориулсан  кодын засварлагч юм. 2015 оны 11-р сарын 18-нд Visual Studio Code-ийн эх код MIT лицензийн дагуу гарч, GitHub дээр тавигдсан.Онцлогуудад нь дибаг хийх, синтакс онцлох, ухаалаг код гүйцээлт, хэсэгчилсэн хэсгүүд, кодын дахин боловсруулалт, суулгагдсан Git зэрэг орно. Хэрэглэгчид загвар, гарын товчлол, тохиргоог өөрчлөх, нэмэлт функц бүхий өргөтгөлүүдийг суулгах боломжтой. Visual Studio Code нь C, C\#, C++, Fortran, Go, Java, JavaScript, Node.js, Python, Rust зэрэг олон төрлийн программчлалын хэлээр ашиглах боломжтой. Энэ нь Blink layout хөдөлгүүр дээр ажилладаг Node.js вэб программуудыг боловсруулахад хэрэглэгддэг Electron framework дээр суурилдаг. Visual Studio Code нь хамгийн түгээмэл программчлалын хэлнүүдийн үндсэн дэмжлэгийг агуулдаг. Энэхүү үндсэн дэмжлэг нь синтакс тодруулах, хаалт тааруулах, код тохируулах боломжтой хэсгүүдийг агуулдаг. Visual Studio Code нь JavaScript, TypeScript, JSON, CSS, HTML-д зориулсан IntelliSense-тэй, мөн Node.js-ийн дибаг хийх дэмжлэгтэй. Нэмэлт хэлний дэмжлэгийг VS Code Marketplace дээр чөлөөтэй ашиглах боломжтой.
\begin{figure}[H]
    \centering
    \includegraphics[scale=0.8]{Figures/chapter1/vscode.png}
    \caption{ Visual Studio Code лого}
    \label{fig: vscode logo}
\end{figure}
\subsection{Кодын менежмент}
%Кодын менежментийг хэрхэн гүйцэтгэсэн, кодыг хаана байршуулж хөгжүүлсэн талаар бичнэ. GitHub, Gitlab etc.




\section{Бүлгийн дүгнэлт}
%-------------------------------------------------------------------------------
\paragraph{} Тус бүлэгт объект илрүүлэлтийн гэж юу болох талаар судалж, зургаас объект илрүүлэлт хийх болон зураг таних гэсэн ойлголтуудыг ялгааг судалж тайлбарласан. Объект илрүүлэлтийн уламжлалт аргад тооцогдох Виола Жонес, HOG болон DPM илрүүлэлтүүд, орчин үеийн архитектуруудад тооцогдох RCNN, FastRCNN, FasterRCNN, SSD, YOLO тэдгээрийг хэрэгжүүлэх процессын талаар судалсан. Уламжлалт аргууд нь ихэвчлэн тодорхой биетийг зураг дээр дүрслэгдэхдээ яаж дүрслэгддэгийг тодорхойлоод тэр тодорхойлолтоо зургаас хайдаг. Жишээ нь хүний царай зурагт дүрслэгдэхдээ дух хэсэг нь цайвар, нүд хэсэг нь бараан, хамар хэсэг нь бараан, хамрын хоёр талаар цайвар харагдана гэж үзээд тийм дүрслэлийг зургаас хайдаг. Харин орчин үеийн аргуудын гол архитектур болох CNN нь зургийн пиксел бүртэй ажиллаж олон төрлийн филтерээр филтердэж, олон төрлийн feature map -уудыг гарган авч тэдгээрээс хамгийн тохиромжтой объектоо тооцоолон гаргадаг.
\paragraph{} Тус бүлэгт ижил төстэй системүүд, системийг хөгжүүлэхэд шаардлагатай технологи болон алгоритмуудыг тайлбарласан. Iashin.ai гэх зөвхөн зурагнаас объект илрүүлэлт хийдэг веб сайт болон TapTapSee гэх зураг болон видеоноос объект илрүүлэлт хийж зураг болон видеоны агуулгыг хэрэглэгчид текст болон аудио хэлбэрээр хүргэдэг мобайл аппликейшныг ижил төстэй системээр сонгон судалж харьцуулалт хийсэн. Объект илрүүлэлт хийх MobileNet V3 - SSD архитектур, сургагдсан моделийг шууд ашиглах боломжийг олгодог OpenCV сан, хөгжүүлэлтийн back-end болон front-end хэсгүүдийг холбоход туслах Python-д зориулсан FastAPI фреймворк. Текстэн мэдээллийг аудио руу хөрвүүлэхэд хэрэглэгдэх Text-to-Speech API, Chimege API. Front-end хөгжүүлэлтэнд ашиглах React фреймворк, AntDesign CSS сан болон PyCharm, VS Code хөгжүүлэлтийн орчнуудыг судалж тайлбарласан.