% Бүлэг 1

\chapter{Хувийн үүлэн хадгалалтын системийн тухай онол, арга зүйн судалгаа} % Бүлгийн нэр
\label{Chapter1} % Энэ бүлэг рүү ишлэл хийх бол \ref{Chapter1} командыг ашигла 

%-------------------------------------------------------------------------------

% Агуулгад ашигласан хэвшүүлэлтийн зарим командын тодорхойлолт
\newcommand{\keyword}[1]{\textbf{#1}}
\newcommand{\tabhead}[1]{\textbf{#1}}
\newcommand{\code}[1]{\texttt{#1}}
\newcommand{\file}[1]{\texttt{\bfseries#1}}
\newcommand{\option}[1]{\texttt{\itshape#1}}

%-------------------------------------------------------------------------------



%-------------------------------------------------------------------------------
\section{Ижил төст системийн судалгаа}
\paragraph{}
Хувийн өгөгдөл хадгалах, синхрончлох системүүд нь сүүлийн жилүүдэд хэрэглэгчийн өгөгдөлд бүрэн хяналттай байх, аюулгүй байдал болон cloud-ын энгийн хэрэглээний туршлагыг хадгалах чиглэлд эрчимтэй хөгжиж байна. Энэ хүрээнд уламжлалт төвлөрсөн cloud системүүд болон peer-to-peer системүүдийг судалгаанд оруулж байна.
\paragraph{}
Nextcloud нь self-hosted cloud storage системүүдийн түгээмэл жишээ бөгөөд сервер төвтэй архитектур ашиглан файл хадгалалт, синхрончлол, хуваалцах болон хамтын ажиллагааны олон төрлийн боломжуудыг нэг дор нэгтгэдэг. Nextcloud нь Universal File Access \cite{nextcloud2018architecture} давхаргаар дамжуулан төрөл бүрийн storage эх үүсвэрүүдийг нэг интерфэйс дор нэгтгэх боломж олгодог. Гэсэн хэдий ч ийм өргөн хүрээтэй боломжууд нь системийг суулгах, тохируулах, удирдах процессыг төвөгтэй болгож, энгийн хэрэглэгчдэд хүндрэл учруулдаг.

\begin{figure}[H]
    \centering
    \includegraphics[scale=0.75]{Figures/chapter1/nextcloud_data_diagram.png}
    \caption{NextCloud системийн өгөгдлийн давхаргын нэгдсэн шийдэл}
    \label{fig:nextcloudDataDiagram}
\end{figure}

\paragraph{}
Нөгөө талаас Syncthing нь peer-to-peer зарчимд суурилсан файл синхрончлолын систем бөгөөд төв сервер шаардалгүйгээр төхөөрөмжүүдийг хооронд нь шууд холбодог. Syncthing нь Local Discovery, Global Discovery, Relay Protocol зэрэг механизмуудыг \cite{syncthingspecifications} ашиглан сүлжээний янз бүрийн орчинд төхөөрөмжүүдийг илрүүлэх, холбох боломжийг олгодог. Мөн бүх өгөгдөл end-to-end encryption (E2EE)-ээр хамгаалагддаг нь аюулгүй байдлын чухал давуу тал юм. Гэвч Syncthing нь олон төхөөрөмж дээр нэг фолдерийг хуулбарлан хадгалах (replication) зарчимд суурилсан тул уламжлалт cloud storage-ийн өгөгдөл нэг газарт төвлөрсөн байдлаар хадгалагдана гэсэн ойлголттой нийцдэггүй. 
\paragraph{}
Мөн хэрэглэгчийн хүлээлтийг тодорхойлогч consumer cloud storage системүүд болох Apple iCloud болон Google Drive нь автомат backup, background upload, минимал тохиргоо зэрэг UX-ийн өндөр стандартыг тогтоосон. Эдгээр системүүд нь хэрэглээний хувьд энгийн боловч өгөгдөл нь гуравдагч талын төв серверт хадгалагддаг.

\subsection{Нийтийн үүлэн хадгалалтын системүүд}
Нийтийн үүлэн хадгалалтын системүүдэд дараах өргөн хэрэглэгддэг платформууд орно:

\begin{enumerate}
    \item Apple iCloud
    \item Google Drive
    \item Microsoft OneDrive
\end{enumerate}

\paragraph{}
Нийтийн үүлэн хадгалалтын системүүд нь автомат синхрончлол, нөөцлөлт зэрэг үйлдлүүдийг хэрэглэгчийн оролцоогүйгээр гүйцэтгэх замаар өндөр түвшний хэрэглээний энгийн байдлыг хангадаг. Гэсэн хэдий ч өгөгдөл нь гуравдагч талын төв серверт хадгалагддаг тул хэрэглэгчийн өгөгдөлд бүрэн хяналттай байх боломж хязгаарлагддаг.

\begin{figure}[H]
    \centering
    \includegraphics[scale=0.75]{Figures/chapter1/cloud_providers.png}
    \caption{Өргөн ашиглагдаж буй үүлэн хадгалалтын үйлчилгээнүүд}
    \label{fig:popularCloudProviders}
\end{figure}


\subsection{Хувийн үүлэн хадгалалтын системүүд}
Self-hosted cloud системүүдийн түгээмэл жишээнүүд:
\begin{enumerate}
    \item NextCloud (opensource)
    \item OwnCloud (opensource)
    \item Seafile (opensource)
\end{enumerate}

\paragraph{}
Self-hosted cloud системүүд нь уламжлалт сервер төвтэй архитектур дээр суурилан ажилладаг бөгөөд хэрэглэгчийн өгөгдлийг өөрийн хяналт дор хадгалах боломжийг олгодог. Эдгээр системүүд нь хэрэглэгчийн интерфэйсийн хувьд харьцангуй боловсронгуй боловч сүлжээний орчин, серверийн тохиргоо, удирдлагын нэмэлт мэдлэг шаарддаг. Иймээс уг төрлийн шийдлүүд нь байгууллагын түвшинд хамтран ашиглахад илүү тохиромжтой бөгөөд хувийн хэрэглээнд ашиглахад техникийн хүндрэл үүсгэх магадлалтай.

\begin{figure}[H]
    \centering
    \includegraphics[scale=0.75]{Figures/chapter1/self_cloud_providers.png}
    \caption{Хувийн сервер дээр ашиглах боломжтой үүлэн хадгалалтын платформууд}
    \label{fig:popularCloudProviders}
\end{figure}


\subsection{Peer-to-Peer өгөгдөл синхрончлолын системүүд}

Peer-to-peer (P2P) зарчимд суурилсан өгөгдөл синхрончлолын системүүд:
\begin{enumerate}
    \item Syncthing (opensource)
    \item Resilio Sync
\end{enumerate}

Эдгээр системүүд нь төхөөрөмж бүрийг давтагдашгүй таних тэмдэг (device identity)-ээр тодорхойлж, төв сервер шаардахгүйгээр хооронд нь шууд холбох боломжийг олгодог. Төхөөрөмжүүдийг илрүүлэхдээ дотоод сүлжээнд LAN discovery механизмыг ашиглах бөгөөд шаардлагатай тохиолдолд relay fallback \cite{syncthingspecifications} шийдлийг ашиглан холболтыг үргэлжлүүлэх боломжтой. Мөн өгөгдөл дамжуулалт нь end-to-end encryption (E2EE)-д суурилсан zero-trust загвараар хамгаалагддаг нь аюулгүй байдлын чухал давуу тал болдог.

Гэсэн хэдий ч эдгээр системүүд нь төхөөрөмжүүдийн хооронд файлыг харилцан хуулбарлан хадгалах (replication) зарчимд суурилдаг тул өгөгдөл нэг төв байршилд хадгалагдах уламжлалт хувийн cloud storage загвараас ялгаатай. Иймээс тэдгээрийн хэрэглээний зорилго нь бүрэн нийцэхгүй боловч дотооддоо ашиглаж буй холболт, илрүүлэлт болон аюулгүй дамжуулалтын технологиуд нь судлан хэрэгжүүлэхэд чухал ач холбогдолтой юм.

\subsection{Харьцуулсан шинжилгээ}

\begin{table}[H]
\centering
\begin{tabular}{|p{0.18\textwidth}|p{0.18\textwidth}|p{0.13\textwidth}|p{0.13\textwidth}|p{0.13\textwidth}|p{0.13\textwidth}|}
\hline
\textbf{Систем} & \textbf{Архитектур} & \textbf{LAN Auto Discovery} & \textbf{Relay Fallback} & \textbf{E2EE} & \textbf{UX энгийн байдал} \\ 
\hline
Nextcloud & Төвлөрсөн self-hosted сервер & Байхгүй & Байхгүй & Хэсэгчлэн & Дунд \\ 
\hline
Syncthing & Peer-to-Peer (P2P) & Байгаа & Байгаа & Байгаа & Харьцангуй төвөгтэй \\ 
\hline
iCloud / Google Drive & Төвлөрсөн нийтийн cloud & Байхгүй & Байхгүй & Хязгаарлагдмал & Өндөр \\ 
\hline
Санал болгож буй систем & Hybrid (Server + Relay) & Байгаа & Байгаа & Байгаа & Өндөр \\ 
\hline
\end{tabular}
\caption{Ижил төст системүүдийн харьцуулалт}
\end{table}

\paragraph{}
Дээрх хүснэгтээс харахад одоо байгаа шийдлүүд нь хэрэглээний энгийн байдал, өгөгдлийн хяналт болон сүлжээний уян хатан байдлын хооронд тодорхой зөрчил (trade-off) үүсгэж байна. Нийтийн үүлэн системүүд нь хэрэглээний хувьд энгийн боловч хэрэглэгчийн өгөгдөлд бүрэн хяналт олгодоггүй. Харин self-hosted болон P2P системүүд нь өгөгдлийн хяналт болон аюулгүй байдлыг сайжруулдаг ч суурилуулалт болон ашиглалтын төвөгшил нэмэгддэг. Иймээс эдгээр системүүд нь хэрэглээний энгийн байдал, аюулгүй байдал болон сүлжээний уян хатан байдлыг зэрэг хангах нэгдсэн шийдлийг бүрэн санал болгож чадахгүй байна.

\section{End-to-End Encryption (E2EE) \& No-Trust Model}

\section{Дамжуулагч серверийн архитектур}

\section{Service Detection in the LAN}

\section{Raspberry Pi-д зориулсан программ хангамж}

\section{Ашиглах технелогиуд}


\section{Бүлгийн дүгнэлт}
%-------------------------------------------------------------------------------
\paragraph{} 