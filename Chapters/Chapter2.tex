% Бүлэг 2

\chapter{... системийн шинжилгээ} % Бүлгийн нэр
\label{Chapter2} % Энэ бүлэг рүү ишлэл хийх бол \ref{Chapter1} командыг ашигла 

\section{Системийн үйл ажиллагааны тухай дэлгэрэнгүй}
Энэхүү систем нь хэрэглэгчээс зурган файлыг авч тухайн зурагт дүрслэгдсэн объектуудын нэрийг олж, тэдгээрийн нэрийг байршилтай нь хамт тухайн зураг дээр нь хүрээлсэн хайрцагт хийж тодотгож харуулдаг объект илрүүлэлт хийдэг веб юм. Объектын илрүүлэлтийг хийж, зургийг хэрэглэгчид буцаан харуулахаас гадна хэрэглэгчид дүрслэгдсэн объектуудыг, илэрсэн тоотой нь хамт текст болон монгол, англи хэл дээрх аудио файл болгон хөрвүүлж өгнө. Хэрэглэгчид утасны дугаар эсвэл өөрийн имейл хаягаар бүртгэл үүсгэж системд нэвтэрвэл тухайн хэрэглэгчийн хэрэглэгчийн бүртгэлээрээ нэвтэрсэн үедээ хийсэн объект илрүүлэлтүүд хадгалагдсан байна. Үүнд объект илрүүлэлт хийсний дараах шинэчлэгдсэн зураг, зурагт хамаарах объект илрүүлэлтийн мэдээллийг агуулсан текст болон аудио файлууд огноотойгоо хамт хадгалагдсан байна. Системийн объект илрүүлэлт нь илрүүлсэн объектоо дараах 80 төрлийн зүйлд хуваан танина. \say{ Хүн, унадаг дугуй, машин, мотоцикл, онгоц, автобус, галт тэрэг, ачааны машин, завь, гэрлэн дохио, галын цорго, зогсох тэмдэг, зогсоолын тоолуур, вандан, шувуу, муур, нохой, морь, хонь, үхэр, заан, баавгай, тахь, анааш, үүргэвч, шүхэр, гар цүнх, зангиа, чемодан, фрисби, цана, снөүбоард, спортын бөмбөг, цаасан шувуу, бейсболын цохиур, бейсболын бээлий, скейтборд, серфингийн самбар, теннисний цохиур, лонх, дарсны шил, аяга,с эрээ, хутга, халбага, аяга, банана, алим, сэндвич, жүрж, брокколи, лууван, хот дог, пицца, донат, бялуу, сандал, буйдан, савтай ургамал, ор, хоолны ширээ, бие засах газар, ТВ монитор, зөөврийн компьютер, хулгана, удирдлага, гар, гар утас, богино долгионы зуух, зуух, талх шарагч, угаалтуур, хөргөгч, ном, цаг, ваар, хайч, бамбарууш, үс хатаагч, шүдний сойз.}
\section{Системийг ашиглах хэрэглэгчид}
\paragraph{} Тус системийг нас, хүйс хамаарахгүйгээр объект илрүүлэлтийг ашиглаж үзэхийг хүссэн, интернэтэд холбогдсон хэн бүр ашиглаж болно. Системийг ашиглаж буй хэрэглэгчдийг бүртгэлтэй болон бүртгэлгүй гэж 2 ангилна. 
\section{Функцийн шаардлага}
Хэрэглэгчийн функцийн шаардлага
 \begin{enumerate}
     \item Хэрэглэгч объект илрүүлэлтэд ашиглахыг хүссэн зургаа веб -д байршуулна.
     \item Аудио файлын хэлийг монгол эсвэл англи хэлний аль нэгээр нь сонгоно.
 \end{enumerate}
Бүртгэлтэй хэрэглэгчийн функцийн шаардлага
 \begin{enumerate}
     \item Хэрэглэгч системд овог нэр, утасны дугаар, имейл хаяг, нууц үг гэсэн талбаруудыг бөглөж бүртгүүлнэ
     \item Имейл хаягаар илгээгдэх ОТР кодоор бүртгэлээ баталгаажуулна
     \item Нэр, утас, имейл хаягийн мэдээллийг агуулсан хэрэглэгчийн бүртгэлийн хэсэгт бүртгэлээ засдаг байна.
     \item Бүртгэлтэй хэрэглэгч өөрийн бүртгэлээ устгаж болно, энэ тохиолдолд бүртгэлтэй хэрэглэгчид хамааралтай бүх файл устана.
     \item Хэрэглэгчийн бүртгэлээр нэвтэрсэн үедээ ашигласан зургуудын объект илрүүлэлтийг агуулсан шинэчилсэн зургууд, тэдгээрт хамаарах дүрслэгдсэн объектуудын мэдээллийг агуулсан текст болон аудио файлууд огноотойгоо хадгалагдсан байна.
     \item Хэрэглэгч өөрт бүртгэлтэй байгаа объект илрүүлэлтийг агуулсан зургууд, тэдгээрт хамаарах текст болон аудио файлтай нь хамт устгаж болно.
 \end{enumerate}
Системийн функцийн шаардлага
 \begin{enumerate}
     \item Хэрэглэгчийн байршуулсан зураг дээр объект илрүүлэлт хийж, объект илрүүлэлтийг агуулсан зургийг хэрэглэгчид буцаан харуулдаг байна.
     \item Объект илрүүлэлтэд илэрсэн объектуудын нэр болон тоо хэмжээг агуулсан текстэн мэдээллийг хэрэглэгчид харуулна.
     \item Объект илрүүлэлтэд илэрсэн объектуудын нэр болон тоо хэмжээг илэрхийлсэн монгол эсвэл англи/хэрэглэгчийн сонголтоор/ хэл дээрх mp4 өргөтгөлтэй аудио файлыг буцаана.
     \item Бүртгэл үүсгэж буй хэрэглэгчийн бүртгэлийг баталгаажуулахын тулд имейл хаягаар OTP код илгээнэ.
 \end{enumerate}
\section{Функцийн бус шаардлага}
 \begin{enumerate}
     \item Хэрэглэгч болон бүртгэлтэй хэрэглэгчийн интерфейс ойлгомжтой, минимал загвартай байна.
     \item Бүртгэлтэй хэрэглэгчийн нууц үг хамгийн багадаа 6 ширхэг том, жижиг үсэг, тоо, тэмдэгтийг агуулсан байна.
     \item 200MB хүртэлх хэмжээтэй файлыг веб -д хуулна.
     \item Веб -д хуулах файл нь jpg, jpeg өргөтгөлтэй байна.
     % \item Бүртгэлд ашиглагдаж буй ОТР нь 1 минутын хүчинтэй хугацаатай байна.
     \item Front-end болон back-end системүүдийн хооронд мэдээлэл солилцох API -н хүсэлт нь multipart-form-data хэлбэртэй байх бол
     хариу нь Application JSON форматтай байна.
     \item Package ба Import statements - Пакеж доторх классуудтай холбоотой нэр үгээр пакежийг нэрлэнэ.
    \item Class Header ба Declaration - Классын нэрийг тухайн класс дотор хийгдэж байгаа үйл ажиллагаатай холбоотой нэр үгээр нэрлэнэ
    \item Method Headers ба Declarations - Метод буюу дэд функцийг тухайн функцийн гүйцэтгэх үйл үгээр нэрлэнэ
    \item Функцийн дээд мөрөнд функцийн тайлбарыг коммент хэлбэрээр бичсэн байна.
    \item Хөгжүүлэлтийн туршид хөгжүүлэлтийн нэг phase бүр дээр нэгжийн тестийг хийдэг байна.
 \end{enumerate}

\section{Юзкейс диаграмм}
\paragraph{} Зураг \ref{fig:usecase} -д дүрслэгдсэн объект илрүүлэлт хийдэг веб системийн юзкейс диаграмм нь бүртгэлтэй болон бүртгэлгүй хэрэглэгч гэсэн тоглогчид болон тэдгээрийн ашиглаж болон нийт 5 юзкейсыг агуулсан. 
\begin{figure}[H]
    \centering
    \includegraphics[scale=0.75]{Figures/chapter2/usecase.pdf}
    \caption{Юзкейс диаграмм}
    \label{fig:usecase}
\end{figure}
\section{Юзкейсийн тодорхойлолт}
\begin{longtable}{|r|p{11.5cm}|}
    \caption{Системд файл хуулах юзкейсийн тодорхойлолт} 
	\label{table:usecase1}\\ \hline
	{Юзкейс:} & {Системд файл хуулах }\\ \hline
	{ID:} & {1 }\\ \hline
	{Үүсгэсэн:} & { Хүслэн}\\ \hline
	{Үүсгэсэн огноо:} & { 2023.03.15 }\\ \hline
	{Үндсэн тоглогч:} & {Бүртгэлтэй хэрэглэгч эсвэл бүртгэлгүй хэрэглэгч }\\ \hline
	{Нэмэлт тоглогч:} & { Байхгүй}\\ \hline
	{Товч тайлбар:} & { Хэрэглэгч объект илрүүлэлтэд ашиглах файлыг веб хуудас руу хуулна.}\\ \hline
	{Өмнөх нөхцөл:} & {\begin{enumerate}[nosep]
	    \item Хэрэглэгч объект илрүүлэлт хийдэг вебийг броузер дээрээ нээсэн байна.
	    \item Веб хуудас руу хуулах гэж файл нь веб хуудас руу хандалт хийж байгаа төхөөрөмж дээр байна.
	\end{enumerate}}\\ \hline
	{Үндсэн урсгал:} & { 
	\begin{enumerate}[nosep]
	    \item Хэрэглэгч веб хуудасны "Байршуулах" button -г дарна.
	    \item Төхөөрөмж дээр байх файлуудыг харуулсан "Open" цонх хэрэглэгчид харагдана.
            \item Хэрэглэгч веб хуудас руу хуулах файлаа сонгоод "Open" цонхны "Open" button -г дарна.
            \item "Open" цонх хаагдана.
	    \item if Файл хуулалт амжилттай болсон бол \
	        \begin{enumerate}[nosep,label*=\arabic*.]
	            \item Вебийн файл байршуулах хэсэгт файл байршсан байна.
	        \end{enumerate}
	    \item if Файл хуулалт амжилтгүй болсон бол
                \begin{enumerate}[nosep,label*=\arabic*.]
	        \item "Амжилтгүй, дахин оролдоно уу + алдааны мессеж" гэсэн контексттэй pop-up -г хэрэглэгчид харуулна.
                \item Файл байршуулах хэсэгт файл байршаагүй байна.
	    \end{enumerate}
	    
	\end{enumerate}}\\ \hline
	{Дараах нөхцөл:} & { \begin{enumerate}
	    \item Системд файл хуулагдсан байна.
	\end{enumerate}}\\ \hline
	{Альтернатив урсгал:} & {Байхгүй. }\\ \hline
\end{longtable}
\begin{longtable}{|r|p{11.5cm}|}
    \caption{Объект илрүүлэлт хийх юзкейсийн тодорхойлолт} 
	\label{table:usecase2}\\ \hline
	{Юзкейс:} & {Объект илрүүлэлт хийх }\\ \hline
	{ID:} & {2 }\\ \hline
	{Үүсгэсэн:} & { Хүслэн}\\ \hline
	{Үүсгэсэн огноо:} & { 2023.03.15 }\\ \hline
	{Үндсэн тоглогч:} & {Бүртгэлтэй хэрэглэгч эсвэл бүртгэлгүй хэрэглэгч }\\ \hline
	{Нэмэлт тоглогч:} & { Байхгүй}\\ \hline
	{Товч тайлбар:} & { Веб хуудсанд байршуулсан файлд объект илрүүлэлт хийх хүсэлт явуулна.}\\ \hline
	{Өмнөх нөхцөл:} & {\begin{enumerate}[nosep]
	    \item Веб хуудсанд объект илрүүлэлтэд ашиглах файлыг хуулсан байна..
	\end{enumerate}}\\ \hline
	{Үндсэн урсгал:} & { 
	\begin{enumerate}[nosep]
	    \item Хэрэглэгч объект илрүүлэлтийн мэдээллийг хүлээн авах хэлээ сонгосон байна.
	    \item Хэрэглэгч "Объект илрүүлэх" button дарна.
	    \item if Объект илрүүлэлт амжилттай бол \
	        \begin{enumerate}[nosep,label*=\arabic*.]
	            \item Объект илрүүлэлт хийсэн файлыг хэрэглэгчид файл байршуулах хэсэгт буцаан харуулна.
                \item Файлд илэрсэн объектуудын нэрийг текст мэдээлэл болгон сонгосон хэл руу хөрвүүлэн "Текст мэдээлэл" хэсэгт харуулна. 
                \item Текст мэдээллийг аудио руу хөрвүүлэн "Аудио мэдээлэл" хэсэгт харуулна.
                \item Объект илрүүлэлтэд ашигласан файл түүний үр дүнг агуулсан текст болон аудио файлуудыг систем бүртгэлтэй хэрэглэгчийн "Хадгалагдсан файлууд" руу нэмнэ.
	        \end{enumerate}
	    \item if Объект илрүүлэлт амжилтгүй болсон бол
                \begin{enumerate}[nosep,label*=\arabic*.]
	        \item "Амжилтгүй, дахин оролдоно уу + алдааны мессеж" гэсэн контексттэй pop-up -г хэрэглэгчид харуулна.
                \item Файл байршуулах хэсэгт файл харагдахгүй, хэлний сонголт арилсан байна.
	    \end{enumerate}    
	\end{enumerate}}\\ \hline
	{Дараах нөхцөл:} & { \begin{enumerate}
	    \item Объект илрүүлэлтийн үр дүнг хэрэглэгч харсан байна.
	\end{enumerate}}\\ \hline
	{Алтернатив урсгал:} & {Байхгүй.}\\ \hline
\end{longtable}

\section{Шинжилгээний класс диаграмм}
\paragraph{} Зураг \ref{fig:sh-class} дээрх системийн шинжилгээний шатны класс диаграмм нь системд үндсэн үүрэг гүйцэтгэх "control" төрөлтэй байж болох 6 классыг тодорхойлж тэдгээрийн хоорондын холбоо хамаарлыг дүрсэлсэн. \say{Үр дүнгийн боловсруулалт} класс нь объект илрүүлэлт хийх хүсэлтийг хүлээн авч, үр дүнг нэгтгэн боловсруулаад буцаах үүрэгтэй класс, \say{Хэлний төрөл} класс нь хөрвүүлэлт хийж болох хэлний төрлүүдийг агуулсан класс, \say{Объект илрүүлэлт} класс нь хэрэглэгчийн хуулсан файлд объект илрүүлэлт хийж илэрсэн объектуудын төрөл болон объект илрүүлэлт хийсэн шинэ файлыг \say{Үр дүнгийн боловсруулалт} класс руу илгээдэг. \say{Хэрэглэгчийн файл} болон \say{Хэрэглэгч} классууд нь хэрэглэгчийн мэдээлэл болон хэрэглэгчийн файлын мэдээлэл зохион байгуулдаг классууд, \say{Бүртгэл} класс нь бүртгэлтэй холбоотой ОТР үүсгэж, шалгаж, бүртгэл үүсгэдэг класс юм.
\begin{figure}[H]
    \centering
    \includegraphics[scale=0.8]{Figures/chapter2/sh-class.pdf}
    \caption{Шинжилгээний класс диаграмм}
    \label{fig:sh-class}
\end{figure}

\section{Шинжилгээний дарааллын диаграмм}
\paragraph{}
Шинжилгээний класс диаграммд тодорхойлсон \say{control} төрлийн 
5 классын операторуудыг ашиглан \say{файл системд хуулах}, \say{объект илрүүлэлт хийх}, \say{бүртгүүлэх}, \say{хэрэглэгчийн файлын мэдээлэл зохион байгуулах}, \say{бүртгэлийн мэдээлэл зохион байгуулах} гэсэн юзкейсүүдэд шинжилгээний шатны дарааллын диаграммыг гаргасан.  
\par \noindent
Бүртгэлтэй хэрэглэгч бүртгэлийн мэдээллээ шинэчлэх эсвэл устгах үед дараах дарааллаар процесс явагдана. Бүртгэлтэй хэрэглэгчийн хүсэлт хэрэглэгчийн мэдээллийг удирдах \say{хэрэглэгч} классаар гүйцэтгэгдэнэ.
Зураг \ref{fig:burtgel-zh-1} -д дээрх дарааллыг агуулсан хэрэглэгчийн бүртгэлийн мэдээлэл зохион байгуулах шинжилгээний дарааллын диаграммыг харуулж байна.
\begin{figure}[H]
    \centering
    \includegraphics[scale=0.6]{Figures/chapter2/burtgel-zh-1.pdf}
    \caption{Хэрэглэгчийн бүртгэлийн мэдээлэл зохион байгуулах шинжилгээний дарааллын диаграмм}
    \label{fig:burtgel-zh-1}
\end{figure}

\par \noindent
Хэрэглэгч системд шинээр бүртгэл үүсгэх хүсэлтийг \say{бүртгэл} класс хүлээн авч системд бүртгэлтэй эсэхийг шалгаад хэрэв бүртгэлгүй бол бүртгэлийн процессыг дуусгаад \say{хэрэглэгч} классаар дамжуулан шинэ хэрэглэгчийг системд бүртгэх хүсэлтийг явуулна. Дээрх процессыг Зураг \ref{fig:burtguuleh-1} дээрх шинжилгээний дарааллын диаграммд дүрсэлсэн.
\begin{figure}[H]
    \centering
    \includegraphics[scale=0.8]{Figures/chapter2/burtguuleh-1.pdf}
    \caption{Бүртгэл үүсгэх шинжилгээний дарааллын диаграмм}
    \label{fig:burtguuleh-1}
\end{figure}
\par \noindent
Зураг \ref{fig:file-zh-1} -н хадгалагдсан файлуудын мэдээллийг зохион байгуулах шинжилгээний дарааллын диаграммд үзүүлсний дагуу системд бүртгэлтэй хэрэглэгч дээр бүртгэгдсэн байгаа объект илрүүлэлт хийсэн файлын мэдээллийг хэрэглэгчид харуулах болон устгах процессыг \say{хэрэглэгчийн файл} класс гүйцэтгэнэ.
\begin{figure}[H]
    \centering
    \includegraphics[scale=0.8]{Figures/chapter2/file-zh-1.pdf}
    \caption{Хадгалагдсан файлуудын мэдээллийг зохион байгуулах шинжилгээний дарааллын диаграмм}
    \label{fig:file-zh-1}
\end{figure}

\par \noindent 
Зураг \ref{fig:fileHuulah-1} -д үзүүлсний дагуу хэрэглэгчийн хуулсан файлыг \say{үр дүнгийн боловсруулалт} класс локал сервер дээрээ хүлээн авч, объект илрүүлэлт хийхэд бэлддэг.
\begin{figure}[H]
    \centering
    \includegraphics[scale=0.8]{Figures/chapter2/fileHuulah-1.pdf}
    \caption{Файл системд хуулах шинжилгээний дарааллын диаграмм}
    \label{fig:fileHuulah-1}
\end{figure}
\par \noindent
Объект илрүүлэлт хийхийн тулд хэрэглэгч \say{үр дүнгийн боловсруулалт} класст файлын мэдээллийг явуулна, дараа нь тус файлд \say{объект илрүүлэлт} класс объект илрүүлэлт хийгээд үр дүнг буцаана. \say{Үр дүнгийн боловсруулалт} класс текст болон аудио мэдээллийг боловсруулаад, бүртгэлтэй хэрэглэгчийн хүсэлт эсэхийг \say{хэрэглэгч} классаар дамжуулан шалгаад, хэрэв тийм бол \say{хэрэглэгчийн файл} класс руу объект илрүүлэлт хийсэн файлын мэдээллийг системд нэмэх хүсэлтийг илгээгээд үр дүнг хэрэглэгчид буцаана, бүртгэлгүй хэрэглэгчийн хүсэлт бол үр дүнг шууд буцаана. Зураг \ref{fig:objectDetection-1} нь дээрх процессыг агуулсан объект илрүүлэлт хийх шинжилгээний дарааллын диаграммыг харуулж байна.
\begin{figure}[H]
    \centering
    \includegraphics[scale=0.6]{Figures/chapter2/objectDetection-1.pdf}
    \caption{Объект илрүүлэлт хийх шинжилгээний дарааллын диаграмм}
    \label{fig:objectDetection-1}
\end{figure}

\section{Үйл ажиллагааны диаграмм}
\paragraph{}
Объект илрүүлэлт хийх веб системийн юзкейс диаграммд тодорхойлогдсон юзкейсүүдийн эхлэлээс төгсгөл хүртэлх үйлдлүүдийг дүрсэлсэн үйл ажиллагааны диаграммуудыг доор харуулав.
\par \noindent
Хэрэглэгч бүртгэлийн мэдээллээ зохион байгуулах үйл ажиллагааг эхлэхийн тулд эхлээд системд нэвтэрсэн байна. Хэрэглэгч мэдээлэл шинэчлэх хэсгийг сонгосноор өөрийн бүртгэлийн мэдээллээ харна. Хэрэглэгч шинэчлэхийг хүссэн мэдээллээ шинэчилж оруулаад шинэчлэх хүсэлт илгээнэ. Систем хүсэлтийг хүлээн аваад мэдээллийг шинэчлээд хэрэглэгчид бүртгэгдсэн эсэх мэдээллийг хүргэнэ, хэрэв бүртгэгдээгүй бол хэрэглэгч хуучин бүртгэлийн мэдээллээ дахин харснаар дээрх процесс давтагдаж болно. Дээрх процессыг Зураг \ref{fig:ua-burtgel-zh} -н хэрэглэгчийн бүртгэлийн мэдээлэл зохион байгуулах үйл ажиллагааны диаграммаас харж болно.
\begin{figure}[H]
    \centering
    \includegraphics[scale=0.65]{Figures/chapter2/ua-burtgel-zh.pdf}
    \caption{Хэрэглэгчийн бүртгэлийн мэдээлэл зохион байгуулах үйл ажиллагааны диаграмм}
    \label{fig:ua-burtgel-zh}
\end{figure}
\par \noindent
Зураг \ref{fig:ua-burtguuleh} -н үйл ажиллагааны диаграмм нь дараах процессыг дүрсэлнэ. Хэрэглэгч бүртгэлийн мэдээллээ бөглөж оруулаад системд бүртгэлтэй эсэхээ шалгуулна, бүртгэлтэй бол хэрэглэгчид \say{бүртгэлтэй} гэсэн мэдээллийг хүргээд үйл ажиллагаа дуусна, бүртгэлгүй бол систем хэрэглэгчийн имейл руу ОТР -г илгээж, баталгаажуулалтыг хүлээнэ. Баталгаажуулалт амжилттай бол бүртгэл үүсэж үйл ажиллагаа дуусна харин амжилтгүй бол хэрэглэгчээс код дахин илгээх эсэхийг лавлаж үйл ажиллагаа ОТР илгээх үйлдэл руу буцна.
\begin{figure}[H]
    \centering
    \includegraphics[scale=0.6]{Figures/chapter2/ua-burtguuleh.pdf}
    \caption{Бүртгэл үүсгэх үйл ажиллагааны диаграмм}
    \label{fig:ua-burtguuleh}
\end{figure}

\par \noindent
Хадгалагдсан файлуудын мэдээллийг зохион байгуулахын тулд хэрэглэгч нэвтэрсэн байна. Дараа нь мэдээлэл харах хэсгээс өөрт байгаа файлуудын мэдээллийг харна. Устгахыг хүссэн файлынхаа ард байрлах \say{устгах} товчийг дарна, систем хэрэглэгчийг устгах эсэхийг лавлаж асууна. Хэрэглэгч файлыг устгана гэдгийг баталгаажуулснаар файл устаж үйл ажиллагаа дуусна. Харин хэрэглэгч лавлах асуултад \say{цуцлах} хэсгийг сонгож файлыг устгахгүйгээр үйл ажиллагааг дуусгаж болно. Зураг \ref{fig:ua-file-zh} дээрх хадгалагдсан файлуудын мэдээллийг зохион байгуулах үйл ажиллагааны диаграмм дээрх процессыг дүрсэлж байна.
\begin{figure}[H]
    \centering
    \includegraphics[scale=0.7]{Figures/chapter2/ua-file-zh.pdf}
    \caption{Хадгалагдсан файлуудын мэдээллийг зохион байгуулах үйл ажиллагааны диаграмм}
    \label{fig:ua-file-zh}
\end{figure}

\par \noindent
Зураг \ref{fig:ua-fileHuulah} -д үзүүлсэн файл системд хуулах үйл ажиллагааны диаграммын дагуу хэрэглэгчийн сонгосон файл эхлээд веб хуудсанд хуулагдана, дараа нь объект илрүүлэх хүсэлттэй хамт илгээгдэж ирснээр back-end сервер дээр хуулагдана.
\begin{figure}[H]
    \centering
    \includegraphics[scale=0.8]{Figures/chapter2/ua-fileHuulah.pdf}
    \caption{Файл системд хуулах үйл ажиллагааны диаграмм}
    \label{fig:ua-fileHuulah}
\end{figure}
\par \noindent
Объект илрүүлэлт хийх үйл ажиллагаа эхлэхийн өмнө файл системд хуулагдсан байна. Файлд эхлээд объект илрүүлэлт хийгээд сонгогдсон хэл дээр текст болон аудио мэдээллийг боловсруулна. Хэрэглэгч системд нэвтэрсэн эсэхийг шалгаад үнэн бол үр дүнг хэрэглэгчийн файлын санд хадгалаад дараа нь үр дүнг хэрэглэгчид буцааснаар үйл ажиллагаа дуусна. Хэрэв хэрэглэгч нэвтэрч ороогүй байвал үр дүнг шууд буцааснаар үйл ажиллагаа дуусгавар болно. Зураг \ref{fig:ua-objectDetection} -д дараах процессыг үйл ажиллагааны диаграммаар дүрсэлсэн.
\begin{figure}[H]
    \centering
    \includegraphics[scale=0.6]{Figures/chapter2/ua-objectDetection.pdf}
    \caption{Объект илрүүлэлт хийх үйл ажиллагааны диаграмм}
    \label{fig:ua-objectDetection}
\end{figure}

\section{Бүлгийн дүгнэлт}
\paragraph{} Энэ бүлэгт системийн үйл ажиллагааг дэлгэрэнгүй тайлбарлаж, системийг ашиглах хэрэглэгчид болон шаардлагуудыг тодорхойлж, юзкейс, юзкейсийн тодорхойлолт болон шинжилгээний шатны класс, дараалал, үйл ажиллагааны диаграммуудыг гаргасан. Эхлээд  системийг хэрэглэх хэрэглэгчдийг тодорхойлоод системд оролцогч тал бүрийн функцийн шаардлагыг тодорхойлж мөн системийн функцийн бус шаардлагыг ч тодорхойлсон. Системд оролцогчдын функцийн шаардлага дээр үндэслэн 2 тоглогчтой 5 юзкейс бүхий юзкейс диаграммыг байгуулж, юзкейс бүрд тодорхойлолт гаргасан. Юзкейсийг хэрэгжүүлж болох шинжилгээний шатны 6 классуудыг гаргаж тэдгээрийн холбоо хамаарлыг тодорхойлсон. Тодорхойлогдсон классууд болон юзкейсийн тодорхойлолтын дагуу дарааллын диаграмм болон үйл ажиллагааны диаграммуудыг гаргаж системийн үйл ажиллагаа, оролцогч талууд, процессын дарааллыг шинжилгээний түвшинд ойлгох боломжийг олгосон.