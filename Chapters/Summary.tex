% Ерөнхий дүгнэлт

\phantomsection
\addchaptertocentry{Ерөнхий дүгнэлт}
\label{Summary} % Энэ бүлэг рүү ишлэл хийх бол
%-------------------------------------------------------------------------------
%	Summary
%-------------------------------------------------------------------------------
\section*{Ерөнхий дүгнэлт}

% Төслийн ерөнхий явц хэр байсан, Төслийн менежментийн хэрхэн гүйцэтгэсэн, кодын менежментыг хэрхэн гүйцэтгэсэн, кодыг хаана байршуулсан, Системийг хаана байршуулж ажиллуулж байгаа, Цаашид системийг хаана байршуулж нийтийн хүртээл болгох, Системийг бодит хэрэглэгч дээр ямар хэмжээнд нэвтрүүлсэн, альфа бета тестүүд ямар хэмжээнд хийгдсэн зэргийг дүгнэлтдээ оруулаарай. 


\paragraph{} “Зургаас объект илрүүлэлт хийх веб хөгжүүлэх нь”  сэдэвтэй дипломын ажлын зорилго нь зурагт дүрслэгдэж буй объектуудыг таних, тэдгээр объектын нэрсийг тодорхойлох, байршлыг дүрслэн харуулах, дүрслэгдсэн объектуудын нэрийг текст болон монгол, англи хэл дээрх аудио хэлбэр рүү хөрвүүлэх, мөн хэрэглэгчид өмнө нь хөрвүүлсэн зураг, тухайн зургийн текст, аудио мэдээллийг хадгалах чадвартай веб сайтыг хөгжүүлэх юм.  Тус веб сайтыг хөгжүүлснээр хэрэглэгчид зурагт агуулагдаж буй объектуудыг ялгаж, таньж мэдэх, илэрсэн объектуудыг нэрээр нь ангилан тоолох, зурагт дүрслэгдсэн жижиг объектуудыг олж харах, харааны бэрхшээлтэй иргэд илэрсэн объектуудын мэдээллийг аудио хэлбэрээр сонссоноор зургийн агуулгыг төсөөлөх гэх мэт боломжийг олгоно. 
\par Системийг хөгжүүлэхийн тулд объект илрүүлэлт гэж юу болох талаар судалж, хэрэглэгч болон системийн шаардлагуудыг тодорхойлж, системийг хөгжүүлэхэд шаардлагатай ЮМЛ -н диаграммуудыг гаргасан. Системийн архитектур нь хэрэглэгчтэй харилцах хэсэг болох front-end болон системийн логик үйл ажиллагааг хангах хэсэг болох back-end гэсэн 2 хэсгээс бүрдэж хоорондоо REST API ашиглан холбогддог байхаар зохиомжилсон. MobileNet SSD V3 архитектурын хөлдөөсөн моделийг авч OpenCV -н DetectionModel классаар объект илрүүлэлт хийх сүлжээ үүсгэж, түүгээрээ дамжуулан зургаас объект илрүүлэлтийг гүйцэтгэж байгаа юм. Текст мэдээллийг аудио руу хөрвүүлэхдээ Python gTTS сан, Chimege API -г ашигласан. Системийн хэрэглэгчтэй харилцах хэсгийг ReactJS фреймворк ашиглан хөгжүүлсэн.
\par Энэхүү дипломын ажил нь зургаас объект илрүүлэлт хийх веб хөгжүүлэхэд ашигласан технологиуд, хэд хэдэн объект илрүүлэлтийн архитектурын талаарх мэдээллийг агуулсан бичиг баримт ба зургаас объект илрүүлэлт хийх веб сайтыг багтаасан болно.

	