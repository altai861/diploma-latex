% Удиртгал
% \phantomchapter{Удиртгал} % Зарим нэг зөвлөмж

\phantomsection
\addchaptertocentry{Удиртгал}
\label{Introduction} % Энэ бүлэг рүү ишлэл хийх бол \ref{Introduction} командыг ашигла 
%-------------------------------------------------------------------------------
%   INTRODUCTION
%-------------------------------------------------------------------------------
% Зорилго зорилт, өмнөх бие даалт харах
    \section*{Удиртгал}
        Өдгөө энгийн ухаалаг утас хэрэглэгч бүр үүлэн хадгалалт буюу Cloud Storage-ийг хэрэглэж байна. Хүн бүрт зураг, бичлэг, ажлын болон хувийн чухал файлууд гэх мэт өгөгдлүүдийг хадгалах хэрэгцээ гардаг бөгөөд түүнийгээ интернетийн тусламжтай хаанаас ч  ашиглах боломж нь орчин үеийн мэдээллийн эрин үед амьдрахтай нь салшгүй нэг хэсэг болсон билээ. Гэвч энэхүү зах зээлийн ихэнх хувийг төвлөрсөн үүлэн хадгалалтын үйлчилгээ үзүүлдэг компаниуд удирдаж байгаа бөгөөд хэрэглэгчдийн өгөгдөл тухайн компанийн сервер дээр хадгалагддаг. Энэхүү хадгалалтын үйлчилгээ нь энгийнээр түрээсийн үйлчилгээ юм. Хэрэглэгчдэд өөрийнхөө өгөгдлийг өөрийн hardware дээр хадгалж, ямар нэг түрээсийн төлбөрөөс ангид, багтаамж хязгаарлалтын асуудлаас ангид гэсэн хэдий ч одоогийн ашигладаг үүлэн хадгалалтын үйлчилгээтэй нь адилхан хэрэглэгчийн сэтгэл ханамжийг өгдөг системийг хөгжүүлэх нь энэ ажлын зорилт билээ. 
    \subsection*{Зорилго}
    Энэхүү төслийн зорилго нь хэрэглэгчийн өөрийн компьютер болон Raspberry Pi зэрэг бага өртөгтэй төхөөрөмж дээр суурилсан хувийн үүлэн хадгалалтын системийг хөгжүүлэхэд оршино. Үүний хүрээнд өгөгдөл хадгалах үндсэн систем, гар утасны хэрэглэгчийн аппликейшн (Client app), болон сүлжээний хязгаарлалтыг шийдвэрлэх дамжуулагч серверийн (Relay server) программ хангамжуудыг иж бүрнээр нь боловсруулна.
    
    \subsection*{Төслийн зорилтууд}
        Дэвшүүлсэн зорилгод хүрэхийн тулд дараах тодорхой зорилтуудыг хэрэгжүүлнэ. Үүнд:
        \begin{enumerate}[nosep]
            \item Хувийн үүлэн хадгалалтын ижил төстэй системүүдийн архитектур болон технологийн харьцуулсан судалгаа хийх;
            \item Нөөц багатай төхөөрөмж (Raspberry Pi г.м)-д зориулсан үүлэн хадгалалтын серверийн программ хангамжийг зохион бүтээх;
            \item Файл удирдах үндсэн үйлдлүүд (upload, download, organize)-ийг дэмжих iOS үйлдлийн системд зориулсан хэрэглэгчийн аппликейшн боловсруулах;
            \item Дотоод сүлжээнд өгөгдлийг хамгийн бага хоцрогдолтой дамжуулах ''LAN-first'' механизмыг нэвтрүүлэх;
            \item Интернэт орчинд олон хэрэглэгч зэрэг хандах боломжтой, зэрэгцээ боловсруулалт (parallel computing) бүхий дамжуулагч серверийн (relay server) архитектурыг шийдвэрлэх;
            \item Өгөгдлийн нууцлал болон аюулгүй байдлыг хангах үүднээс төгсгөлийн цэг хоорондын (End-to-End) шифрлэлтийг хэрэгжүүлэх.
        \end{enumerate}

    \subsection*{Судлах үндэслэл}
        Бүх төхөөрөмжүүд холбогдсон өнөөгийн нийгэмд үүлэн хадгалалт (cloud storage)--ийг ашиглах хүний тоо 2.3 тэрбум гэсэн тоог 2025 оны байдлаар давсан байна. Энэ нь томоохон үүлэн хадгалалтын үйлчилгээ үзүүлдэг Google, Apple, Microsoft гэсэн компаниудад хүмүүсийн ихэнх өгөгдөл хадгалагдаж байгаа гэсэн үг юм. Энэ хувь хүмүүсийн чухал дата өөр газар байх нь зарим талаараа эрсдэлийг дагуулдаг. 
        Мөн энэхүү үүлэн хадгалалтын үйлчилгээнүүд сарын болон жилийн төлбөр хураамж авдаг. Бага хэмжээний хадгалах боломжийг үнэгүй өгөөд түүнээс дээшээ төлбөр авдаг. Google Drive (Google One) 100GB storage-ийг жилийн 20 доллараар зардаг. 200GB-ийг 30 доллар. 2TB-ийг 100 доллар гэсэн ханшаар зарж байна. Өөрийн их хэмжээний файлыг хадгалахын тулд заавал төлбөртэй хувилбар луу шилжих шаардлага гардаг. Мөн өөрийн их хэмжээний өгөгдлийг гуравдагч талд өгч байгаа нь зарим хүнд сул тал болох хандлагатай.

    \subsection*{Судлагдсан байдал}
        Хувийн үүлэн хадгалалтын системийн чиглэлээр дэлхий дахинд Nextcloud, Seafile, Syncthing зэрэг нээлттэй эхийн төслүүд өргөнөөр судлагдаж, хөгжүүлэгдсээр ирсэн. Эдгээр системүүд нь хэрэглэгчийн өгөгдлийн нууцлалыг хангах үндсэн зорилготой боловч Raspberry Pi зэрэг нөөц багатай төхөөрөмж дээр ажиллахад системийн ачаалал ихсэх, тохиргоо хийхэд хэт нарийн мэргэжлийн мэдлэг шаарддаг зэрэг дутагдалтай талууд ажиглагддаг.

        Сүлжээний хандалтын хувьд NAT traversal болон Reverse Proxy-д суурилсан шийдлүүдийг ашигладаг боловч олон хэрэглэгч зэрэг хандах үед дамжуулах хурд удаашрах асуудал тулгардаг. Иймд энэхүү төслөөр бага чадалтай төхөөрөмжид оновчтой (lightweight) ажиллах серверийн бүтэц болон зэрэгцээ боловсруулалт бүхий дамжуулагч серверийн архитектурыг ашиглан дээрх асуудлуудыг шийдвэрлэхээр зорьсон.
        
   \subsection*{Шинэлэг тал}
    \textbf{Хувийн үүлэн хадгалалтын системийн шинэлэг тал:} 
        \begin{enumerate}[itemsep=0mm]
            \item \textbf{Өгөгдлийн бүрэн эзэмшил (Data Governance)}: Хэрэглэгчийн өгөгдөл гуравдагч талын серверт бус, зөвхөн хэрэглэгчийн өөрийн техник хангамжид хадгалагдаж, нууцлал нь бүрэн хяналтад байна.
            \item \textbf{Бага нөөцөд зориулсан шийдэл (Resource-Efficient)}: Raspberry Pi болон хуучин компьютер зэрэг бага чадалтай төхөөрөмжүүдэд зориулсан хөнгөн (lightweight) архитектуртай.
            \item \textbf{Хялбар тохиргоо (Zero-Config Networking)}: Дамжуулагч серверийн тусламжтайгаар сүлжээний нарийн тохиргоо (Port forwarding, Static IP) шаардахгүйгээр шууд ашиглах боломжтой.
        \end{enumerate}

    \subsection*{Технологийн ач холбогдол}
         Энэхүү систем нь технологийн хувьд дараах ач холбогдолтой:
         \begin{enumerate}[itemsep=0mm]
            \item \textbf{Сүлжээний бие даасан байдал}: Дамжуулагч серверийн шийдэл нь интернэтийн дурын орчноос (Public IP-гүй байсан ч) саадгүй холбогдох боломжийг олгоно.
            \item \textbf{Зардал хэмнэлт}: Үүлэн хадгалалтын тогтмол төлбөрийг халж, байгаа нөөцөө ашиглан эдийн засгийн хэмнэлт гаргана.
            \item \textbf{Аюулгүй байдлын шинэ стандарт}: End-to-End шифрлэлтийг ашигласнаар дамжуулагч сервер хүртэл өгөгдлийг унших боломжгүй болж, нууцлалын өндөр түвшинд хүрнэ.
        \end{enumerate}

    \subsection*{Хамрах хүрээ}
    Тус төсөл нь дараах хүрээг хамарна:
    \begin{itemize}[nosep]
        \item Өөрийн өгөгдлийн нууцлалыг эрхэмлэдэг хувь хүн болон гэр бүлийн хэрэглэгчид;
        \item Өндөр өртөг бүхий үүлэн үйлчилгээнээс татгалзаж, өөрийн нөөцийг ашиглах хүсэлтэй технологи сонирхогчид;
    \end{itemize}