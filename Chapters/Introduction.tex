% Удиртгал
% \phantomchapter{Удиртгал} % Зарим нэг зөвлөмж

\phantomsection
\addchaptertocentry{Удиртгал}
\label{Introduction} % Энэ бүлэг рүү ишлэл хийх бол \ref{Introduction} командыг ашигла 
%-------------------------------------------------------------------------------
%   INTRODUCTION
%-------------------------------------------------------------------------------
% Зорилго зорилт, өмнөх бие даалт харах
    \section*{Удиртгал}
        Өдгөө энгийн ухаалаг утас хэрэглэгч бүр үүлэн хадгалалт буюу Cloud Storage-ийг хэрэглэж байна. Хүн бүрт зураг, бичлэг, ажлын болон хувийн чухал файлууд гэх мэт өгөгдлүүдийг хадгалах хэрэгцээ гардаг бөгөөд түүнийгээ интернетийн тусламжтай хаанаас ч  ашиглах боломж нь орчин үеийн мэдээллийн эрин үед амьдрахтай нь салшгүй нэг хэсэг болсон билээ. Гэвч энэхүү зах зээлийн ихэнх хувийг төвлөрсөн үүлэн хадгалалтын үйлчилгээ үзүүлдэг компаниуд удирдаж байгаа бөгөөд хэрэглэгчдийн өгөгдөл тухайн компанийн сервер дээр хадгалагддаг. Энэхүү хадгалалтын үйлчилгээ нь энгийнээр түрээсийн үйлчилгээ юм. Иймд хэрэглэгч өөрийн техник хангамжийг ашиглан өгөгдлөө хадгалах замаар түрээсийн төлбөр болон багтаамжийн хязгаарлалтаас ангид байх, үүний зэрэгцээ одоогийн үүлэн үйлчилгээнүүдтэй ижил түвшний хэрэглээний сэтгэл ханамжийг өгч чадах системийг хөгжүүлэх нь энэхүү ажлын гол зорилго юм. 
    \subsection*{Зорилго}
    Энэхүү төслийн зорилго нь хэрэглэгч өөрийн эзэмшлийн техник хангамжид суурилсан, гуравдагч этгээдээс хамааралгүй, бие даасан хувийн үүлэн хадгалалтын системийг хөгжүүлэхэд оршино. Тус систем нь өгөгдлийн өмчлөл болон нууцлалыг бүрэн хангахын зэрэгцээ дотоод болон гадаад сүлжээний орчинд төхөөрөмжүүд хооронд өгөгдлийг саадгүй, аюулгүй дамжуулах нэгдсэн шийдлийг бүрдүүлнэ.
    
    \subsection*{Төслийн зорилтууд}
        Дэвшүүлсэн зорилгод хүрэхийн тулд дараах тодорхой зорилтуудыг хэрэгжүүлнэ. Үүнд:
        \begin{enumerate}[nosep]
            \item Хувийн үүлэн хадгалалтын ижил төстэй системүүдийн архитектур болон технологийн харьцуулсан судалгаа хийх;
            \item Нөөц багатай төхөөрөмж (Raspberry Pi г.м)-д зориулсан үүлэн хадгалалтын серверийн программ хангамжийг зохион бүтээх;
            \item Файл удирдах үндсэн үйлдлүүд (upload, download, organize)-ийг дэмжих iOS үйлдлийн системд зориулсан хэрэглэгчийн аппликейшн боловсруулах;
            \item Дотоод сүлжээнд өгөгдлийг хамгийн бага хоцрогдолтой дамжуулах ''LAN-first'' механизмыг нэвтрүүлэх;
            \item Интернэт орчинд олон хэрэглэгч зэрэг хандах боломжтой, зэрэгцээ боловсруулалт (parallel computing) бүхий дамжуулагч серверийн (relay server) архитектурыг шийдвэрлэх;
            \item Өгөгдлийн нууцлал болон аюулгүй байдлыг хангах үүднээс төгсгөлийн цэг хоорондын (End-to-End) шифрлэлтийг хэрэгжүүлэх.
        \end{enumerate}

    \subsection*{Судлах үндэслэл}
        Өнөөгийн дижитал шилжилтийн эрин үед үүлэн хадгалалт (Cloud Storage) нь хэрэглэгчдийн өдөр тутмын хэрэгцээ болж, 2025 оны байдлаар дэлхий даяар ашиглагчдын тоо 2.3 тэрбумыг давсан үзүүлэлттэй байна. Гэвч энэхүү өсөлтийг дагаад дараах хоёр үндсэн асуудал тулгарч байна:

        \begin{enumerate}
            \item \textbf{Өгөгдлийн нууцлал ба хараат байдал:}
            Зах зээлийн дийлэнх хувийг Google, Apple, Microsoft зэрэг томоохон компаниуд эзэлж байгаа нь хэрэглэгчдийн хувийн мэдээлэл гуравдагч этгээдийн серверт төвлөрөхөд хүргэж байна. Энэ нь өгөгдөл эзэмших эрх (data ownership) болон нууцлалын аюулгүй байдалд эрсдэл дагуулдаг бөгөөд хэрэглэгч өөрийн датаг хянах боломжийг хязгаарладаг.
            \item \textbf{Эдийн засгийн зардал ба хязгаарлагдмал нөөц:}
            Төвлөрсөн үүлэн үйлчилгээнүүд нь хадгалах багтаамжаас хамаарсан тогтмол (сарын эсвэл жилийн) төлбөр шаарддаг. Тухайлбал, Google One үйлчилгээний 100GB-аас 2TB хүртэлх багтаамж нь жилийн 20-100 долларын хооронд хэлбэлзэж байна. Их хэмжээний өгөгдөл хадгалах шаардлагатай хэрэглэгчдийн хувьд энэ нь урт хугацаандаа эдийн засгийн дарамт болдог.
        \end{enumerate}

        Иймд хэрэглэгч өөрт байгаа техник хангамжийн нөөц боломжийг ашиглан, ямар нэгэн нэмэлт төлбөргүйгээр өөрийн өгөгдлийг бүрэн хянах, аюулгүй хадгалах боломжийг бүрдүүлсэн хувийн үүлэн системийг хөгжүүлэх нь технологийн болон практик ач холбогдолтой юм.

    \subsection*{Судлагдсан байдал}
        Хувийн үүлэн хадгалалтын системийн чиглэлээр дэлхий дахинд Nextcloud, Seafile, Syncthing зэрэг нээлттэй эхийн төслүүд өргөнөөр судлагдаж, хөгжүүлэгдсээр ирсэн. Эдгээр системүүд нь хэрэглэгчийн өгөгдлийн нууцлалыг хангах үндсэн зорилготой боловч Raspberry Pi зэрэг нөөц багатай төхөөрөмж дээр ажиллахад системийн ачаалал ихсэх, тохиргоо хийхэд хэт нарийн мэргэжлийн мэдлэг шаарддаг зэрэг дутагдалтай талууд ажиглагддаг.

        Сүлжээний хандалтын хувьд NAT traversal болон Reverse Proxy-д суурилсан шийдлүүдийг ашигладаг боловч олон хэрэглэгч зэрэг хандах үед дамжуулах хурд удаашрах асуудал тулгардаг. Иймд энэхүү төслөөр бага чадалтай төхөөрөмжид оновчтой (lightweight) ажиллах серверийн бүтэц болон зэрэгцээ боловсруулалт бүхий дамжуулагч серверийн архитектурыг ашиглан дээрх асуудлуудыг шийдвэрлэхээр зорьсон.
        
   \subsection*{Шинэлэг тал}
    \textbf{Хувийн үүлэн хадгалалтын системийн шинэлэг тал:} 
        \begin{enumerate}[itemsep=0mm]
            \item \textbf{Өгөгдлийн бүрэн эзэмшил (Data Governance)}: Хэрэглэгчийн өгөгдөл гуравдагч талын серверт бус, зөвхөн хэрэглэгчийн өөрийн техник хангамжид хадгалагдаж, нууцлал нь бүрэн хяналтад байна.
            \item \textbf{Бага нөөцөд зориулсан шийдэл (Resource-Efficient)}: Raspberry Pi болон хуучин компьютер зэрэг бага чадалтай төхөөрөмжүүдэд зориулсан хөнгөн (lightweight) архитектуртай.
            \item \textbf{Хялбар тохиргоо (Zero-Config Networking)}: Дамжуулагч серверийн тусламжтайгаар сүлжээний нарийн тохиргоо (Port forwarding, Static IP) шаардахгүйгээр шууд ашиглах боломжтой.
        \end{enumerate}

    \subsection*{Технологийн ач холбогдол}
         Энэхүү систем нь технологийн хувьд дараах ач холбогдолтой:
         \begin{enumerate}[itemsep=0mm]
            \item \textbf{Сүлжээний бие даасан байдал}: Дамжуулагч серверийн шийдэл нь интернэтийн дурын орчноос (Public IP-гүй байсан ч) саадгүй холбогдох боломжийг олгоно.
            \item \textbf{Зардал хэмнэлт}: Үүлэн хадгалалтын тогтмол төлбөрийг халж, байгаа нөөцөө ашиглан эдийн засгийн хэмнэлт гаргана.
            \item \textbf{Аюулгүй байдлын шинэ стандарт}: End-to-End шифрлэлтийг ашигласнаар дамжуулагч сервер хүртэл өгөгдлийг унших боломжгүй болж, нууцлалын өндөр түвшинд хүрнэ.
        \end{enumerate}

    \subsection*{Хамрах хүрээ}
    Тус төсөл нь дараах хүрээг хамарна:
    \begin{itemize}[nosep]
        \item Өөрийн өгөгдлийн нууцлалыг эрхэмлэдэг хувь хүн болон гэр бүлийн хэрэглэгчид;
        \item Өндөр өртөг бүхий үүлэн үйлчилгээнээс татгалзаж, өөрийн нөөцийг ашиглах хүсэлтэй технологи сонирхогчид;
    \end{itemize}