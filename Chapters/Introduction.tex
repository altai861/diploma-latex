% Удиртгал
% \phantomchapter{Удиртгал} % Зарим нэг зөвлөмж

\phantomsection
\addchaptertocentry{Удиртгал}
\label{Introduction} % Энэ бүлэг рүү ишлэл хийх бол \ref{Introduction} командыг ашигла 
%-------------------------------------------------------------------------------
%   INTRODUCTION
%-------------------------------------------------------------------------------
% Зорилго зорилт, өмнөх бие даалт харах
    \section*{Удиртгал}
        Олон нийтийн анхаарлыг татаж, амьдралынх нь салшгүй хэрэглээ болсон системүүдийн томоохон давуу тал нь тасалдалгүй тогтвортой байдал байдаг ба энэ нь хэрэглэгчдэд итгэл найдвар, сэтгэл ханамжийн таатай байдлыг төрүүлдэг. Системийн гацалт, доголдол, алдаа гэх мэт асуудлууд нь ижил төстэй системүүдтэй өрсөлдөх давуу талыг сулруулж, хэрэглэгчдээ алдахад нөлөөлдөг. Цаг үеийн техник технологийн хувьслаар хэрэглэгчдийн хүлээлт, шаардлага нь эрчимтэй өсөж байгаа ба эдгээрийг шийдэж чадсан нь ижил төстэй системүүдийн чиг хандлагын тодорхойлж, хүмүүсийн хэрэглээг дараагийн шинэ төвшинд хүргэдэг болсон.
           Нөгөө талаас тотгортой байдлыг хангадаг системүүд нь олон хэрэглэгчдийн хандалтын ачааллыг хүлээж авахын тулд нөөцийн хувираалал болон түүний зардлын асуудалтай нүүр тулгардаг. Мөн техник болон програм хангамжийн засвар үйлчилгээгээ алдаагүйгээр хурдан шуурхай автоматжуулан хүргэх нь хоёр дахь том асуудал болдог. Эдгээр асуудлуудыг програм хангамжийн хөгжүүлэлтийн автоматжуулах Gitlab-CI/CD ашиглан шийдвэрлэх нь энэхүү ажлын зорилго болно.
    \subsection*{Зорилго}
    Энэхүү төслийн ажлын зорилго нь программ хангамжийн тогтвортой байдал, хандалтын ачааллыг тэнцвэржүүлэх зэрэг асуудлыг Gitlab-CI/CD ашиглан автоматжуулж шийдвэрлэхэд оршино.
    
    \subsection*{Зорилт}
        Уг зорилгод хүрэхийн тулд дараах зорилтуудыг дэвшүүлж байна. Үүнд:
        \begin{enumerate}[nosep]
            \item Gitlab-CI/CD технологийг судлах;
            \item Ижил төстэй системийн судалгаа хийх;
            \item Gitlab-CI/CD онцлог давуу тал, төрлүүдийг судлах ;
            \item Судалгааны туршилтад ашиглах Docker-технологийн талаар судлах;
            \item Судалгаанд тулгуурлан Maven-project ыг Gitlab-CI/CD ашиглан туршиж үзэх;
            \item Туршилт дээр шинжилгээ, дүгнэлт хийх;
        \end{enumerate}

    \subsection*{Судлах үндэслэл}
        continuous integration/continous delivery(CI/CD) нь дараах давуу талуудыг өгдөг тул өргөн хэрэглээтэй болсон ба ПХ хөгжүүлэлтийг автоматжуулах Gitlab-CI/CD технологийг судлах үндэслэл болсон.
        \begin{enumerate}[nosep]
            \item Кодын чанарыг сайжруулна
            \par CI/CD pipeline нь test автоматжуулалтыг санал болгодог бөгөөд,хөгжүүлэгчид кодын асуудлын талаар бараг бодит цаг хугацаанд мэдэх боломжтой.
            
            \item Програмын шинэ хувилбарыг маш хурдан түгэээнэ
            \par Нэгдсэн CI/CD pipeline нь програм хангамжийн шинэчлэлтийг түгээх хурдыг маш ихээр нэмэгдүүлнэ.
            
            \item CI/CD pipeline: Автоматжуулалт нь зардлыг бууруулдаг
            \par Хүн ямар ч үед програм хангамж боловсруулах үйл явцад хөндлөнгөөс оролцох шаардлагагүй, цаг хугацаа, улмаар мөнгө хэмнэгддэг. Тийм ч учраас автоматжуулалт нь DevOps-ын амжилттай туршлагын үндэс суурь болдог. CI/CD нь дамжуулалт, эх кодын удирдлага, хувилбарын хяналтын систем, байршуулах механизм, мэдээжийн хэрэг test-ын ихэнхийг автоматжуулдаг.
        \end{enumerate}
        

    \subsection*{Судлагдсан байдал}
        CI/CD гэдэг нь /Continuous Integration and Continuous Delivery / гэсэн үгийн товчлол юм.(CI) гэдэг нь автоматаар шалгах сонголт нь юм. Товчхондоо бол source code байгаа repo-руу code commit хийх болгонд тест код ажилладаг. Ингэснээр системийг алдаагүй ажиллаж байгаа, бусад системүүдтэй асуудалгүй интеграци хийгдэж байгааг шалгах боломжтой болдог,(CD) гэдэг нь CI-н дараачийн алхам бөгөөд хэрэглэгчийн гар дээр бүтээгдэхүүнийг хүргэх тэр процессыг автоматжуулахыг хэлдэг. Тестээ давчихсан бүтээгдэхүүнээ package-лаад бэлдээд тавьчихдаг.CI/CD-г ажиллуулах олон технологи байдаг ба тэдний дотроос Gitlab болон Jenkins түгээмэл ашиглагддаг.Монголд 2014,2015 оны үеэс Gitlab CI/CD  ашиглагдаж эхэлсэн гэх мэдээллийг олон жил хөгжүүлэгч хийж байгаа хэд хэдэн ахлах хөгжүүлэгчдээс мэдээлэл авсан.CI/CD нь орчин үед DevOps инженэрүүдйин байнгийн ажилдаг технологи болоод байна.   
        
    \subsection*{Шинэлэг тал}
    \textbf{Gitlab CI/CD шинэлэг тал:} 
        \begin{enumerate}[itemsep=0mm]
            \item Олон гар ажиллагааны процессыг автоматжуулдаг;
            \item Нэмэлт үйлчилгээнүүд;
            \item Хувийн хяналтын систем;
            \item \textbf{Ease of configuration}-Олон төрлөөр тохиргооны custimzed script-үүд бичдэг;
            \item \textbf{Source code security}-Аюулгүй байдал нь код хаана байх, түүнд хэн хандаж болохыг хянаж болдог.
            \item \textbf{Pipeline automation}-CI/CD pipeline-аар програмуудыг автоматаар илрүүлэх, бүтээх, турших, байршуулах, хянах боломжтой Auto DevOps хэмээх функцийг агуулдаг. 
            \item \textbf{DevOps maturity feedback}-GitLab нь хэрэглэгчдэд DevOps-д зориулсан CI/CD дамжуулах хэрэгслийг хэрхэн хэрэгжүүлсэнээс хамаарч оноо өгдөг. Энэ оноо нь багууд жишээлбэл DevOps-ын чадавхийг хаана өргөжүүлэх, хөгжүүлэгчид GitLab функцийг зөв ашиглаж байгаа эсэхийг тодорхойлоход тусалдаг.
            \item \textbf{Deployment scheduling}-GitLab-ийн CI/CD дамжуулах шугамын хуваарийн тусламжтайгаар та тодорхой салбарыг байршуулах цагийг зааж өгч болно. Та ирээдүйд тодорхой хугацаанд эсвэл хүссэн хугацаандаа дахин дахин автомат хуваарийг тохируулах боломжтой
        \end{enumerate}
    \subsection*{Технологийн ач холбогдол}
         Дээрх шинэлэг талуудаас Gitlab CI/CD нь дараах ач холбогдлуудыг авчирна.
                 \begin{enumerate}[itemsep=0mm]
            \item Test,Deployment,шинэчлэлтийг хянах, автоматжуулах
            \item Бүтээгдхүүнийг хэрэглэгчийн гар дээр түргэн шуурхай хүргэх
            \item Олон дамжуулах хоолой зохион байгуулах
            \item Нөөц болон програмыг бодит цаг хугацаанд нь тохируулан аппликешнуудыг тестлэн, шинэчлэх
        \end{enumerate}

        
    \subsection*{Хамрах хүрээ}
    Үр ашигтай, тогтвортой, ухаалаг зардлаар системээ хөгжүүлэх гэж буй хувь хүн, баг, байгууллага хэн ч ашиглах боломжтой юм.Gitlab-ийн үндсэн сервэр,эсвэл хувь байгуулгын ямарч сэрвэр дээр суулган ашиглах боломжтой. 